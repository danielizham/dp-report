\documentclass[12pt]{article}

% For document layout and formatting
\usepackage[a4paper, vmargin=2.54cm,
hmargin=2.54cm]{geometry}
\usepackage{lineno}
\linespread{1.1} % 1.6 for double spacing and 1.3 for one-and-a-half spacing
\usepackage[utf8]{inputenc}
\usepackage[english]{babel}
\usepackage{subfiles}
\usepackage[toc,page]{appendix}

% For fonts and symbols
\usepackage{times}
\usepackage{amsmath}
\usepackage{amssymb}
\usepackage{amsthm}
\usepackage{amsfonts}
\usepackage{siunitx}

% For references
\usepackage[backend=biber, style=ieee]{biblatex}
\addbibresource{references/refs-daniel.bib}
\addbibresource{references/refs-abdulrahman.bib}
\addbibresource{references/refs-bahri.bib}
\usepackage{cleveref} % [capitalise,noabbrev] options
\usepackage{csquotes}

% For equations
\usepackage{extramarks}
\usepackage{float}
\usepackage{cancel} % to use diagonal strikethrough to zero
\usepackage{physics} % to use absolute and norm (for vectors) symbols
\usepackage{array}
\usepackage{tabularx}
\usepackage{calc}
\usepackage{microtype}
\usepackage[plain]{algorithm}
\usepackage{algpseudocode}

% For plotting
\usepackage{pgfplots}
\usepackage{pgfplotstable}
% \pgfplotstabletypeset{search path={tables}}
\pgfplotsset{
    compat=newest,
    % ignore zero/.style={%
    %     #1ticklabel={\ifdim\tick pt=0pt 0 \else\pgfmathprintnumber{\tick}\fi}
    %     },
    yticklabel={\ifdim\tick pt=0pt 0 \else\pgfmathprintnumber{\tick}\fi},
    table/search path={tables,graphs},
} % Allows to place the legend below plot
\usepgfplotslibrary{units} % Allows to enter the units nicely

% For figures
\usepackage{graphicx}
\graphicspath{ {./images/} }
\usepackage{wrapfig}
\usepackage{tikz}
\usetikzlibrary{automata,positioning,calc,fit,matrix}

% For tables
\usepackage{booktabs}
\usepackage{multirow}
\usepackage{diagbox}
\usepackage{makecell}
\usepackage{caption}
\captionsetup[table]{skip=5pt}

% For better lists
\usepackage{enumerate}

% For dummy text
\usepackage{blindtext}
\usepackage{lipsum}

\addto\captionsenglish{%
  \renewcommand{\contentsname}%
    {Table of Contents}%
}

\renewcommand{\appendixtocname}{Appendix}
\renewcommand{\appendixpagename}{Appendix}

\linenumbers

\begin{document}

\subfile{sections/title}

\pagenumbering{roman}

\addcontentsline{toc}{section}{Declaration}
\section*{Declaration}
\subfile{sections/declaration}
\newpage

\addcontentsline{toc}{section}{Abstract}
\section*{Abstract}
\subfile{sections/abstract}
\newpage

\addcontentsline{toc}{section}{Acknowledgment}
\section*{Acknowledgment}
\subfile{sections/acknowledgment}
\newpage

\tableofcontents
\newpage

\addcontentsline{toc}{section}{\listfigurename}
\listoffigures
\addcontentsline{toc}{section}{\listtablename}
\listoftables
\newpage

\pagenumbering{arabic}

\section{Introduction and Motivation}
\subfile{sections/introduction}

\section{Background and Related Work}
\subfile{sections/background}

\section{Requirements Analysis}
\subfile{sections/requirements}

\section{Proposed Solution}
\subfile{sections/solution}

\section{Proof of Concept}
\subfile{sections/poc}

\section{Market Research and Business Viability}
\subfile{sections/market}

\section{Project Plan}
\subfile{sections/plan}

\section{Short Guide}
\subfile{sections/guide}

\newpage

\printbibliography

\newpage

\begin{appendices}
    \subfile{sections/appendix}
\end{appendices}

\end{document}
