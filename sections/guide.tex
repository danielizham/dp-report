\documentclass[../main.tex]{subfiles}

\begin{document}

Please read the guides available online about the right way
to write {\LaTeX} such as how to include a math symbol in
text (e.g. $x$ not x) and a proper noun with all capitals
(e.g. \textsc{sql} not SQL).

Below are examples of different constructs in a report. You
can copy-paste and change the content. For more information,
refer to the relevant package manual in \textsc{ctan}.

\subsection{Abbreviations}

To add an abbreviation (e.g. \textsc{uav}), 
append the following line 
in the list of abbreviations portion in main.tex:
\begin{center}
    \verb|\newacronym{uav}{\textsc{uav}}{unmanned aerial vehicle}|
\end{center}

\noindent
To use the abbreviation, there are 3 ways to do so:
\begin{enumerate}
    \item In a normal case: \verb|\gls{uav}|
    \item\label{item:plural} For its plural form: \verb|\glspl{uav}|
    \item\label{item:beginning} In the beginning of a sentence: \verb|\Gls{uav}|
    \item A combination of cases \ref{item:plural} 
        and \ref{item:beginning}: \verb|\Glspl{uav}|
\end{enumerate}

\noindent
For example:
\begin{quotation}
    \noindent
    An \gls{uav} has many unique features. 
    \Glspl{uav} have been used in many different applications.
\end{quotation}

\subsection{Figure}

\begin{figure}[htb] 
    \centering
    \includegraphics[width=0.8\textwidth]{arch.png} 
    \caption{The arch linux logo} \label{fig:arch-linux} 
\end{figure}

\subsection{Equations}

\begin{align}
        E_p &= mgh = mg(x_f - x_i) \label{potential}
        \\
        E_k &= E_t + E_r \nonumber
        \\
        E_t &= \frac{1}{2} mv^2 \label{translational}
        \\
        E_r &= \frac{1}{2} I \omega^2 \label{rotational}
        \\
        I &= \frac{1}{2} M R^2 \label{inertia}
        \\
        \omega &= \frac{v}{r} \nonumber
        \\
        E_k &= \frac{1}{2} mv^2 +  \frac{1}{2} I \left( \frac{v}{r}\right) ^2 \label{kinetic}
\end{align}
where~$E_p$ is the potential energy, $E_k$ the kinetic
energy, $E_t$ the translational energy and~$E_r$
the rotational energy.

\begin{align*}
        \pdv{E_p}{m} &= \pdv{m}(mgh)
        \\
                          &= gh
                          \\
        \pdv{E_p}{g} &= \pdv{g}(mgh)
        \\
                          &= mh
        \\
        \pdv{E_p}{h} &= \pdv{h}(mgh)
        \\
                          &= mg
\end{align*}

\subsection{Simple table}

\begin{table}[H]
    \centering
    \caption{Slope, intercept and their uncertainties}
    \label{tab:slope}
    \begin{tabular}{*{4}c}
        \toprule
        \multicolumn{2}{c}{Slope} &
        \multicolumn{2}{c}{Intercept (\si{\joule})} \\
        Value & Error & Value & Error \\
        \cmidrule(r){1-2} \cmidrule(l){3-4}
        1.0933 & 0.0300 & 0.0148 & 0.0157 \\

        \bottomrule
    \end{tabular}
\end{table}

\subsection{Table from a csv file}

\begin{table}[H]
    \begin{center}
        \caption{Translational and rotational energies.}
        \label{tab:rot-energies}
        \pgfplotstabletypeset[
        multicolumn names, % allows to have multicolumn names
        col sep=comma, % the seperator in our .csv file
        display columns/0/.style={
            column name=$m$, % name of first column
        column type={S},string type},  % use siunitx for formatting
        display columns/1/.style={
            column name=$v_m$,
        column type={S},string type},
        display columns/2/.style={
            column name=$E_t$,
        column type={S},string type},
        display columns/3/.style={
            column name=$\delta E_t$,
        column type={S},string type},
        display columns/4/.style={
            column name=$E_r$,
        column type={S},string type},
        display columns/5/.style={
            column name=$\delta E_r$,
        column type={S},string type},
        every head row/.style={
            before row={\toprule}, % have a rule at top
            after row={
                \si{\kilo\gram} & 
                \si{\meter\per\second} & \si{\joule} & \si{\joule} & \si{\joule} & \si{\joule} \\ % the units seperated by &
            \midrule} % rule under units
        },
        every last row/.style={after row=\bottomrule}, % rule at bottom
        ]{rotational_energies.csv} % filename/path to file
    \end{center}
\end{table}

\subsection{Graph from a csv file}
\begin{figure}[H] 
    \begin{center}
        \begin{tikzpicture}
    \begin{axis}[
        width=\dimexpr\linewidth-6pt, % Scale the plot to \linewidth
        grid=major, % Display a grid
        grid style={dashed,gray!30}, % Set the style
        title={Potential Versus Kinetic Energies},
        xlabel=Kinetic Energy{,} $E_k$, % Set the labels
        ylabel=Potential Energy{,} $E_p$,
        x unit=\si{\joule}, % Set the respective units
        y unit=\si{\joule},
        xmin=0.0001, % xmax=3,
        ymin=0, % ymax=3,
        % ignore zero=x,
        xtick distance=0.1,
        % ytick distance=0.05,
        legend pos=north west,
        % legend style={at={(0.5,-0.2)},anchor=north}, % Put the legend below the plot
        x tick label style={rotate=90,  % Display labels sideways
            anchor=east,
            /pgf/number format/precision=1,
            /pgf/number format/fixed zerofill,
            /pgf/number format/fixed,
        },
        y tick label style={/pgf/number format/precision=1,
            /pgf/number format/fixed zerofill,
            /pgf/number format/fixed,
        },
        ]
        \addplot [
            blue,
            mark=*,
            mark size=1.20,
            only marks,
            error bars/.cd,
            x dir=both,
            y dir=both,
            x explicit,
            y explicit,
            error bar style={black,semithick}
            ]
            % add a plot from table; you select the columns by using the actual name in
            % the .csv file (on top)
            table[x=kinetic,x error=kinetic error,y=potential,y error=potential error,col sep=comma] {potential_v_kinetic.csv};
        \addlegendentry{$E_p \text{ vs.\@ } E_k$};

        \addplot [no markers, red]
            table[x=kinetic,y={create col/linear regression={y=potential}},col sep=comma] {potential_v_kinetic.csv};

        \addlegendentry[/pgf/number format/precision=6]{
                $ \quad y = \pgfmathprintnumber[precision=4, fixed zerofill]{\pgfplotstableregressiona} \, x
                \pgfmathprintnumber[precision=4, print sign]{\pgfplotstableregressionb}, R^2 = 0.9977 $
            };
    \end{axis}
\end{tikzpicture}

        \caption{The relationship between potential and
        kinetic energies.} \label{fig:pot-kin-energies}
    \end{center}
\end{figure}

\subsection{Citations}

\begin{itemize} 
    \item \textbf{in-text citation}: use
        \verb|\cite{dirac}| to produce \cite{dirac} or
        \verb|\textcite{dirac}| to produce \textcite{dirac}
    \item \textbf{citation in parentheses}:
        \verb|\parencite{knuthwebsite}| produces 
        \parencite{knuthwebsite} (for IEEE, this has no 
        difference to the \verb|\cite{}| command
        above.)
\end{itemize}

\subsection{Cross-references}

Label using suitable names with the following format: figure 
\verb|\label{fig:<name>}|, tables \verb|\label{tab:<name>}|,
sections \verb|\label{sec:<name>}| and equations\\ 
\verb|\label{eq:<name>}|.

Then when cross-referencing, use
\verb|\cref{<type>:<name>}|\\
(or \verb|\Cref{<type>:<name>}| when used at the beginning 
of a sentence)

\end{document}
