\documentclass[../main.tex]{subfiles}

\begin{document}

\subsection{Background}

\blindtext

\subsection{Related work}

% Part 1

% Part 2
% {{{

% Background 
% {{{{
The second important concept of the project is the computer simulation.
Simulation is a cost-effective, time-saving, flexible 
and safe way to experiment with a drone
at the expense of a reduction in accuracy compared to the real world.
The research realm has shown that the use of simulation 
is highly attractive in \gls{drl} studies with drones.
This is because \gls{drl} involves
making gradual improvements to a model based on 
repeated cycles of experience, and computer simulation 
allows these iterations to be carried out cheaply.

According to the literature review, 
the combination of Gazebo, \gls{ros}
and \textsc{px4} are the most widely used software stack 
for the \gls{sitl} development
because they are open-source. 
In contrast, Sphinx and Olympe, which are used in this project, 
are closed-source.
Although open-source software 
allows for full control and the flexibility to tinker
since the source code is freely available,
it is less stable and time-consuming to debug
when there is an error in the source code.
Another conclusion from the literature is that 
the transfer from simulation to the real world,
which this project also aims to accomplish,
is lacking in the studies reviewed.
As a result, it is not possible to comment on 
how the simulation findings effectively translate
to the physical environment.
% }}}}

% Related work
% {{{{ Zhou2020
% Topic sentence
The use of simulation makes rapid experiments in realistic settings 
and iterative \gls{uav} designs possible, 
both of which are important in \gls{ai} training. 
% Explanation
\citeauthor{Zho20} demonstrates this by % or \textcite{Zho20}
using a combination of the open-source 3D dynamic simulator Gazebo
and the autopilot system \textsc{px4}.
Through this, they avoided the time-consuming steps of 
carrying out physical experiments
and adjusting parameters according 
to the environmental settings~\cite{Zho20}.
Thanks to the simulation, 
the authors were also able to propose a generic
framework to integrate the \gls{dqn} algorithm into 
the simulated \gls{uav} environment~\cite{Zho20}.
In our work, the same Gazebo physics engine
simulation software is used, and \gls{drl} is similarly trained
for the high-level control of the \gls{uav}. 
However, the authors used the \gls{ros}-\textsc{px4} as the controller 
while in our work, the Olympe program is used.
The main criticism against this paper is that the operating system,
which was Ubuntu 16.04, and 
the \gls{ros} version, which was Kinetic Kame, 
were old and no longer supported 
even though the paper was written in 2020.
Nevertheless, the explanation and the flowchart illustrating the 
Q-learning in the context of drone control are instructive 
for our work going forward.
% }}}}

% {{{{ Walker2019
% Topic sentence
The time-saving benefit and the ease of experimentation 
afforded by the use of computer simulation are further emphasized 
when studying uncertain environments.
% Explanation
Dealing with an unknown environment for search and navigation applications,
\citeauthor{Wal19} used simulation to train a \gls{uav}
to solve a local planning problem
by framing the problem as 
a \gls{pomdp}
using continuous action spaces~\cite{Wal19}.
Similar to the previous paper, the \gls{ros}-\textsc{px4} stack 
and the Gazebo 
simulation software were used compared to Olympe and Sphinx 
in this project~\cite{Wal19}.
In addition, both the paper and this project study path planning 
with \gls{drl} but our work uses it for target visitation 
in an obstacle-free environment 
while the authors used it for searching and navigation
in both obstacle-free and non-obstacle-free environments.
However, the use of a \gls{uav} indoor by the authors as an application 
does not leverage the unique features of \glspl{uav}, 
but it is a good starting point 
and easier to implement in the real world 
when the outdoor flight is restricted.
A useful lesson that this paper provides for our project
is the use of the open \gls{ai} gym in creating the \gls{uav} environment
resulting in clearer abstraction in the codebase
for the training process.
% }}}}

% {{{{ Garcia2020
% Topic sentence
Yet another \gls{uav} research-related applications 
that profits from the use of computer simulation 
is the testing of new sensors on the \gls{uav}.
% Evidence 1
\citeauthor{Gar20} argues that the future of \glspl{uav}
relies on the use of advanced sensors and 
the ease of analyzing their functions
in real operational conditions~\cite{Gar20}.
% Response
To demonstrate such viability, they connected a \gls{lidar} sensor
to a PixHawk flight controller and tested the improvement
that the new sensor provided
in the application of navigation and obstacle avoidance.
Importantly, prior to that, they used QGroundControl and the \textsc{px4}
platforms to analyze the addition of a \gls{lidar} sensor
on a simulated 3DR Iris \gls{uav}.
Unlike our work, the authors' focus for using the simulation
was on sensor integration and not \gls{drl} 
which did not feature in the paper. 
The main criticism of this work is that 
the sensor is simulated without noise
when in the real world, the data captured
by the sensor is invariably noisy.
Although the objectives between the authors' and our work
are different, it is still very helpful to learn from
the extensive software stack and architecture guide 
presented by the authors.
% }}}}

% }}}

% Part 3

\blindtext

\end{document}

