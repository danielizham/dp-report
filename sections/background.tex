\documentclass[../main.tex]{subfiles}

\begin{document}

\subsection{Background}

\blindtext

\subsection{Related work}

% Part 1

% Part 2
% {{{

% Background 
% {{{{
The second important concept of the project is the computer simulation.
Simulation is a cheap and safe way to experiment with the drone
at the expense of reduction in accuracy compared to the real world.
The research realm shows that the use of simulation is highly attractive
in deep reinforcement learning studies with drones.
This is because Deep Reinforcement Learning involves
making gradual improvement to a model based on 
repeated cycles of experience, and computer simulation 
allows these iterations to be carried out cheaply.

According to the literature reviewed, 
Gazebo and ROS-PX4 are the most widely used software stack 
for the software-in-the-loop (\textsc{SITL}) development
because the pieces of software are open-source. 
In contrast, Sphinx and Olympe are used in this project 
which are closed-source.
Although open-source software in this case 
allows for full control and the flexibility to tinker,
it is less stable and time-consuming to debug
when there is a bug in the source code.
Another conclusion from the literature is that 
the transfer from simulation to the real world,
which this project aims to accomplish,
is lacking in the studies 
which means that the actual effectiveness is missing.
% }}}}

% Related work
% {{{{ Zhou2020

% Topic sentence
The use of simulation makes rapid experiments in realistic settings 
and iterative UAV designs possible which is important in AI training. 
% Explanation
\citeauthor{Zho20} demonstrates that by 
using a combination of the open-source 3D dynamic simulator Gazebo
and the autopilot system PX4
thus ridding them from having to carry out physical experiments
and adjusting parameters according to the environmental settings,
both of which are time-consuming \cite{Zho20}.
Thanks to the simulation, 
the authors also were able to propose a generic
framework to integrate the DQN algorithm into 
the UAV environment \cite{Zho20}.
Compared to the current work, the same Gazebo physics engine
simulation software is used and DRL is similarly trained
for the high level control of the UAV. 
However, the authors used the ROS-PX4 as the controller 
while in our work, the Olympe is used for the Anafi drone.
A main criticism against this paper is that the operating system,
which was Ubuntu 16.04, and 
the ROS version, which was ROS Kinetic Kame, 
were old and no longer supported 
even though the paper was written in 2020.
Nevertheless, the explanation and the flowchart illustrating the 
Q-learning in the context of the drone control are instructive 
for our work going forward.

Hi there \textcite{Zho20}
% }}}}

% {{{{ Walker2019
% Topic sentence
The time-saving benefit of computer simulation is emphasized 
when studying uncertain environments.

% Explanation
Hi there \cite{Wal19}
% }}}}

% {{{{ Garcia2020
Hi there \cite{Gar20}
% }}}}

% }}}

% Part 3

\blindtext

\end{document}

