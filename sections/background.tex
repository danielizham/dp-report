\documentclass[../main.tex]{subfiles}

\begin{document}

\subsection{Background}

\blindtext

\subsection{Related work}
% Part 1

% Part 2

% Part 3
% {{{
% Background 
% {{{{
		The third and final concept of the project is hardware realization for drone visits.
		The hardware part is essential in the implementation in the real world
		, where the simulation sometimes strays from the truth.
		There is a lack of hardware implementations in the field of research regarding drones with \gls{drl}
		 ,and most of the research papers focus on the simulations.
		According to the literature review, 
		the \gls{cnn} models were used in the majority of the papers for object detection. 
		Also, the controller boards and custom drone kits were used instead of commercial drones.
		Those kits give the researcher and user more flexibility since the drone is customizable 
		in hardware and software.But in our design, we will use a commercial drone so that we focus
		 on the \gls{drl}, not the actual drone build process. 

% }}}}

% Related work
% {{{{ Khan2021
% Topic sentence
	The use of a microcomputer with Quadcopter \glspl{uav} and autopilot software will help in the hardware implementation part.
	% Explanation
	\citeauthor{Khan21} used the drone in the agriculture field to spray 
	pesticides and monitor the crops. Unlike our work, the drone was limited to specific
	boundaries and fixed targets such as crops.
	They used a Raspberry Pi microcomputer board attached to the drone
	, which will handle two different operations. Firstly, it will control the drone using an open-source
	 software called Arducopter autopilot which will handle the trip of the drone and autonomous 
	 flight option. The second operation is to deal with the Intel neural computer stick 2, 
	 which will deploy the \gls{cnn} model and deal with the computation part~\cite{Khan21}.
	Although this work is close to ours, there are some differences, 
	one of them is using a custom drone which is not considered since we are limited in time.
	Since we will use the Anafi drone, the Olympe program will take control of the drone, which will
	 be installed on the Raspberry Pi. Finally, using \gls{cnn} only is not enough without \gls{drl} 
	 which makes the drone more intelligent and accurate.


% }}}}

% {{{{ Wang2018
% Topic sentence
	A helpful example that uses a commercial drone with an onboard computer and uses SDK with image processing techniques.
	% Explanation
	The hardware architecture in \citeauthor{Wang18} work for this paper includes a DJI 
	commercial drone and an onboard computer called manifold, which is from the same manufacturer.
	 Also, onboard sensors like camera,\gls{gps} and inertial sensor are included. Finally,
	 external battery for the manifold computer and Wi-Fi adapter that is used for connection 
	 between the drone and the onboard computer. This hardware architecture is inspirational
	 , and our design is somehow close to it with minor changes in the onboard computer and 
	 without the existence of the sensors.Image and video processing techniques were used, such as 
	 segmentation to keep detecting moving targets was presented in ~\cite{Wang18}.
	For the navigation part, they used predetermined waypoints related to historical path cost. 
	However, in our work, probability and mobility patterns will be used to guess the target's location.
% }}}}

% {{{{ Zhao2018
% Topic sentence
    An embedded system connected and attached to the \gls{uav} and uses mobility pattern recognition, which shortens response time and saves transmission bandwidth. 
	% Explanation 
	\citeauthor{Zhao18} work used a quadrotor \gls{uav} supported with \gls{gps} module and a Pix 
	Hawk flight controller. The power sources in the architecture were two lithium batteries, 
	one for the drone and one for the embedded system. The system uses	NVIDIA Jetson development
	 kits which give enough computing power for the processing and communication between the flight 
	 controller and the system. The Jetson board is connected to the flight controller using serial 
	 communication while connected to the ground controller using Wi-Fi. Communication tools and protocols used in \citeauthor{Zhao18} work 
	  will help us to determine the best way to communicate between the development 
	  board and the drone without any delay or interference ~\cite{Zhao18}. 


% }}}}

% }}}



\end{document}

