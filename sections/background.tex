\documentclass[../main.tex]{subfiles}

\begin{document}

\subsection{Background}

\blindtext

\subsection{Related work}

% Part 1

% Part 2
% {{{

% Background 
% {{{{
The second important concept of the project is the computer simulation.
Simulation is a cheap and safe way to experiment with the drone
at the expense of reduction in accuracy compared to the real world.
The research realm shows that the use of simulation is highly attractive
in deep reinforcement learning studies with drones.
This is because Deep Reinforcement Learning involves
making gradual improvement to a model based on 
repeated cycles of experience, and computer simulation 
allows these iterations to be carried out cheaply.
% TODO: add the problem the simulation is trying to solve

According to the literature reviewed, 
Gazebo and ROS-PX4 are the most widely used software stack 
for the software-in-the-loop (\textsc{SITL}) development
because the pieces of software are open-source. 
In contrast, Sphinx and Olympe are used in this project 
which are closed-source.
Although open-source software in this case 
allows for full control and the flexibility to tinker,
it is less stable and time-consuming to debug
when there is a bug in the source code.
Another conclusion from the literature is that 
the transfer from simulation to the real world,
which this project aims to accomplish,
is lacking in the studies 
which means that the actual effectiveness is missing.
% }}}}

% Related work
% {{{{ Zhou2020
% Topic sentence
The use of simulation makes rapid experiments in realistic settings 
and iterative UAV designs possible which is important in AI training. 
% Explanation
\citeauthor{Zho20} demonstrates that by % or \textcite{Zho20}
using a combination of the open-source 3D dynamic simulator Gazebo
and the autopilot system PX4
thus ridding them from having to carry out physical experiments
and adjusting parameters according to the environmental settings,
both of which are time-consuming \cite{Zho20}.
Thanks to the simulation, 
the authors also were able to propose a generic
framework to integrate the DQN algorithm into 
the UAV environment \cite{Zho20}.
Compared to the current work, the same Gazebo physics engine
simulation software is used and DRL is similarly trained
for the high level control of the UAV. 
However, the authors used the ROS-PX4 as the controller 
while in our work, the Olympe is used for the Anafi drone.
A main criticism against this paper is that the operating system,
which was Ubuntu 16.04, and 
the ROS version, which was ROS Kinetic Kame, 
were old and no longer supported 
even though the paper was written in 2020.
Nevertheless, the explanation and the flowchart illustrating the 
Q-learning in the context of the drone control are instructive 
for our work going forward.

% }}}}

% {{{{ Walker2019
% Topic sentence
The time-saving benefit and the ease of experimentation 
afforded by the use of computer simulation is furthur emphasized 
when studying uncertain environments.
% Explanation
Dealing with an unknown environment for search and navigation applications,
\citeauthor{Wal19} used simulation to train a UAV 
to solve a local planning problem
by framing the problem as a 
partial oberservable markov decision process (POMDP)
using continuous action spaces.
Similar to the previous paper, the ROS-PX4 stack and the Gazebo 
simulation software was used compared to Olympe and Sphinx 
in this project.
In addition, both these paper and project study path planning 
with DRL but our work uses it for target visitation 
in an obstacle-free environment 
while the authors used it for searching and navigation
in obstacle-free and non-obstacle-free environments.
However, the use of a UAV indoor by the authors as an application 
does not leverage the unique features of UAVs 
but it is a good starting point 
and easier to implement in the real world 
when outdoor flight is restricted.
A useful lesson that this paper provides for our project
is the use of open AI gym in creating the UAV environment
resulting in clearer abstraction in the codebase
for the training process.
% }}}}

% {{{{ Garcia2020
% Topic sentence
Yet another UAV research-related applications 
that profits from the use of computer simulation 
is the testing of new sensors on the UAV.
% Evidence 1
\citeauthor{Gar20} argues that the future of UAVs
relies on the use of advanced sensors and 
the ease of analysing their functions
in real operational conditions~\cite{Gar20}.
% Response
To demonstrate such viability, they connected a LIDAR sensor
to a PixHawk flight controller and tested the improvement
that the new sensor provided
in the application of navigation and obstacle avoidance.
Importantly, prior to that, they used QGroundControl and the PX4
platforms to analyse the addition of a LIDAR sensor
on a simulated 3DR Iris UAV.
Unlike our work, the authors' focus in using the simulation
was on sensor integration and not DRL, 
which did not feature in the paper at all. 
A main criticism of this work is that 
the sensor is simulated without noise
when in the real world, the data captured
by the sensor is invariably noisy.
Although the objectives between the authors' and our work
are different, it is still very helpful to learn from
the extensive software stack and architecture guide 
presented by the authors.
% }}}}

% }}}

% Part 3

\blindtext

\end{document}

