\documentclass[../main.tex]{subfiles}

\begin{document}

\subsection{Background}



\subsection{Related work}

% Part 1

% Part 2

% Part 3

The main idea here is to make the drones autonomous and intelligent in support of target detection features. The drone was limited to specific boundaries and fixed targets such as crops in the agriculture field [7].
In our work, the \gls{uav} will scan mobile targets intelligently and will guess their location.
 A microcontroller was used to control the drone and execute commands just like our work, but we will use ANAFI SDK for ANAFI drones, not custom ones presented in [7], [9]. 
Image and video processing techniques were used, such as segmentation to keep detecting moving targets was presented in [8].
For the navigation part in [8], they used predetermined waypoints related to historical path cost. However, in our work, probability and mobility patterns will be used. What these papers need are some intelligent algorithms and power and time consideration. Here comes the role of \gls{drl}, which will make the system more intelligent and efficient.




\end{document}

