\documentclass[../main.tex]{subfiles}

\begin{document}

\subsection{Background}

\blindtext

\subsection{Related work}

% Part 3

% Background 
% {{{{
The third and final concept of the project is hardware realization for drone visits. The hardware part is essential in the implementation in the real world, where the simulation sometimes strays from the truth.
There is a lack of hardware implementations in the field of research regarding drones with \gls{drl} ,and most of the research papers focus on the simulations.

According to the literature review, 
the \gls{cnn} models were used in the majority of the papers for object detection. 
Also, the controller boards and custom drone kits were used instead of commercial drones. Those kits give the researcher and user more flexibility since the drone is customizable in hardware and software. But in our design, we will use commercial drones so that we focus on the \gls{drl}, not the actual drone build process. 

% }}}}

% Related work
% {{{{ Khan2021
% Topic sentence
Quadcopter \gls{uav} with arducopter autopilot installed on raspberry pi microcomputer board and intel neural computer stick 2.
% Explanation
\citeauthor{Khan21} used the drone in the agriculture field to spray pesticides and monitor the crops. Unlike our work, the drone was limited to specific
boundaries and fixed targets such as crops.
They used a Raspberry Pi microcomputer board attached to the drone, which will handle two different operations. Firstly, control the drone using an open-source Software called arducopter autopilot which will handle the trip of the drone and autonomous flight option. The second operation is to deal with the Intel neural computer stick 2, which will deploy the \gls{cnn} model and deal with the computation part~\cite{Khan21}.
Although this work is close to ours, there are some differences, 
one of them is using a custom drone as it is not considered since we are limited in the time.
Since we will use the Anafi drone, we will use the Olympe to control the drone, which will be installed on the raspberry pi, finally using \gls{cnn} only is not enough \gls{drl} will make the drone more intelligent and accurate.


% }}}}

% {{{{ Wang2018
% Topic sentence
An example of commercial drone usage with onboard computer that transforms drones into autonomous and the usage of data and image filtration.
% Explanation
The hardware architecture in \citeauthor{Wang18} work for this paper includes a DJI commercial drone and an onboard computer called manifold, which is from the same manufacturer. Also, onboard sensors like camera,\gls{gps} and inertial sensor are included, finally an external battery for the manifold computer and Wi-Fi adapter that is used for connection between the drone and the onboard computer. This hardware architecture is Inspirational, and our design is somehow close to it with minor changes in the onboard computer and the existence of the sensors.   
Image and video processing techniques were used, such as segmentation to keep detecting moving targets was presented in ~\cite{Wang18}.
For the navigation part, they used predetermined waypoints related to historical path cost. However, in our work, probability and mobility patterns will be used to guess the target's location.
% }}}}

% {{{{ Zhao2018
% Topic sentence
Use embedded system connected and attached on the \gls{uav} for image processing and mobility pattern recognition, which shortens response time and saves transmission bandwidth. 
% Explanation 
\citeauthor{Zhao18} work used quadrotor \gls{uav} supported with \gls{gps} module and Pix Hawk flight controller. The power sources in the architecture were two lithium batteries, one for the drone and one for the embedded system. The system uses	NVIDIA Jetson development kits which give enough computing power for the processing and communication between the flight controller and the system. The Jetson board is connected to the flight controller using serial communication while connected to the ground controller using Wi-Fi. In communication, section will help us in our work to determine the best way to communicate between the development board and the drone without any delay or interference ~\cite{Zhao18}. 


% }}}}

% }}}



\end{document}

