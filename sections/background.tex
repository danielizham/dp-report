\documentclass[../main.tex]{subfiles}

\begin{document}

\subsection{Background}

\blindtext

\subsection{Related work}
% summary table 

% Part 1
One of the essential ideas of the project is navigating and tracking the objects while minimizing the required time to detect all targets. Various methods and approaches were studied and implemented in previous research papers with different constraints and goals in mind. The methodology and algorithm in each paper was different as some of them used AI related algorithms while others relied on heavy mathematical calculations to determine the best path.
In paper \citeauthor{hua21}, the main idea was to propose a navigation algorithm that enables each UAV to determine its own movement locally and track pedestrians (mobile targets), it focused on multiple drones to cover a specific area. \citeauthor{pen21} took the advantage of DRL to develop an online path planning algorithm based on double deep Q-learning network (DDQN). The constraints were to minimize the energy consumption of the UAV, the objects on the ground were not stationary and were following a Gauss-Markov movement pattern. Author \citeauthor{hua20} aimed to propose a reactive real-time sliding mode control algorithm to navigate a team of UAVs (UAS). The area was divided into multiple sub-areas using the Voronoi partitioning technique, each drone was responsible for a sub-area, he implemented his ideas for both types of tergets, stationary targets and mobile. \\
All the mentioned papers presented their solutions using different simulation software. However, none of them was implemented in the real-world which questions the reliability of the algorithms.

% Part 2

% Part 3

%
%%%%% Start the summary table here, it is not working currently, I've been trying to solve it 
%\newpage
%	\begin{table}[hbt!]%hbt to stop the table from going to the next page.		
%	\begin{tabular}{ | c | P{4cm} | c | c | P{2.4cm} | P{2.5cm} |}			
%		\hline
%		\textit{Ref No.} & \textit{Main Objective} & \textit{Simulation} & \textit{Real-world} & \textit{Single or Multi Drone} & \textit{Technique} \\ \hline
%		
%		~\cite{huang-1-2020-Reactive}  & Minimize the time to scan targets and propose a reactive real-time sliding mode control algorithm to navigate a team of UAVs. & Yes
%		& No & Single and Multi & Complex Math Calculations (NO AI) \\ \hline
%		
%		~\cite{peng-1-2021-Deep}  & To develop an online path planning algorithm using DRL, minimize energy consumption and maximize offloaded data bits. & Yes & No & Single & Deep Reinforcement Learning \\ \hline
%		
%		~\cite{huang-1-2021-navigating} & To navigate a team of unmanned aerial vehicles (UAVs) to monitor groups of ground pedestrians and deliver a high-quality surveillance & Yes & Yes & Multi & Complex Math Calculations (NO AI) \\ \hline
%		
%		~\cite{zhou-2-2020-An-efficient}  & To adopt existing simulators and propose a general framework to incorporate DQN algorithm with the UAV simulation environment & Yes & \_ & \_ & DRL \\ \hline
%		
%		~\cite{walker-2-2019-a-deep-reinforcement}  & To present the UAV indoor search problem two separate problems, a global planning problem and a local one, create a model to test the application of modern DRL approaches to the two problems. & Yes & No & Single & Deep Reinforcement Learning \\ \hline
%		
%		~\cite{garcia-2-2020-simulation-in-real} & To use available software for mission definition and execution in UAVs based on PixHawk flight controller and peripherals in integrating and evaluating a LIDAR sensor & Yes & Yes & Single & Sensors \\ \hline
%		
%		~\cite{wang-3-2018-Research-UAV-Detection} 	& To build a deep learning model that verifies the efficiency of DL for object detection & Yes & Yes & Single & Deep Learning and CNN \\ \hline
%		
%		~\cite{wang-3-2018-online} & Create a powerful online drone based moving target detection for site inspection and a violation detection tool & No & Yes & Single & Deep Learning and CNN \\ \hline
%		
%		~\cite{khan-3-2021-real}  & To build integrate a deep learning model in UAVs to spray pesticides and monitor the crops.& Yes & Yes & Single & Deep Learning and CNN \\ \hline
%	\end{tabular}
%	\caption{Literature review table summary}
%\end{table}	\label{tab: literature-review-summary} 

\end{document}

