\documentclass[../main.tex]{subfiles}

\begin{document}

\subsection{Background}

\blindtext

\subsection{Related work}
% summary table 

% Part 1
	\subsubsection{Drone's visitation of mobile targets}

One of the essential ideas of the project is navigating and tracking the objects while minimizing the required time to detect all targets. 
This is a critical idea as the efficiency of any algorithms or project is always judged based on time and resource usage.
Various methods and approaches were studied and implemented in previous research papers with different constraints and goals in mind.
As most of the drones have a low battery life, and consequently low flying time. The time constraint was a major limitation for most of the previous research papers. 
Some authors considered other factors such as the quality of data communication (i.e. throughput, latency, etc) between the command/control system and the drone.
The methodology and algorithm in each paper was different as some of them used AI related algorithms while others relied on heavy mathematical calculations to determine the best path.

% #1
In paper ~\cite{hua20}, the authors \citeauthor{hua20} addressed a problem of autonomous navigation of unmanned aerial vehicles. 
They proposed a reactive real-time sliding mode control algorithm that navigates a team of communicating \glspl{uav}.
The drones were equipped with ground-facing video cameras, towards moving targets. 
Furthermore, they adopted the Voronoi partitioning (VP) technique to reduce the range of movement for each \gls{uav} and minimize the revisit time of each target as each drone was responsible for a sub-area.
They also ran extensive computer simulations using matlab. Their simulations were tried on one \gls{uav} and multiple \glspl{uav},
and they also considered the case where of an uneven ground. 
Their main findings where that the use of VP technique leads to more computation burden, but can considerably reduce the target revisit time.

% #2

In paper ~\cite{pen21}, authors \citeauthor{pen21} studied the idea of taking advantage of \glspl{uav} in order to increase the network coverage and execute computing tasks offloaded from multiple devices. 
The constraints were to minimize the energy consumption of the \gls{uav}, and maximize the amount of offloaded bits. 
Objects on the ground were not stationary and were following a Gauss-Markov movement pattern. Their approach was to apply \gls{drl} to develop an online path planning algorithm based on double deep Q-learning network (DDQN).
Simulations were made to prove the viability of their idea and the effectiveness of the \gls{drl}-based path planning algorithm.

% #3
In a third paper, \citeauthor{hua21} focused on navigating a team of \glspl{uav} equipped with cameras to monitor groups of ground pedestrians or vehicles that were moving with a bounded speed but an unknown pattern. 
The surveillance is supposed to be from the shortest distances possible. 
They proposed an algorithm in which the \gls{uav} is able to its movement locally with some help from a central station.
Simulations were done to prove the algorithm's efficiency. However, the specific simulation software was not mentioned. 

All the mentioned papers presented their solutions using different simulation software. However, none of them was implemented in the real-world which questions the reliability of the algorithms.



% Part 2
%{{{

% Background 
% {{{{
% Part 3



\end{document}

