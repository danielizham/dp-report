\documentclass[../main.tex]{subfiles}

\begin{document}

\subsection{Background}

\blindtext

\subsection{Related work}
% summary table 

% Part 1
One of the essential ideas of the project is navigating and tracking the objects while minimizing the required time to detect all targets. Various methods and approaches were studied and implemented in previous research papers with different constraints and goals in mind. The methodology and algorithm in each paper was different as some of them used AI related algorithms while others relied on heavy mathematical calculations to determine the best path.
In paper \citeauthor{hua21}, the main idea was to propose a navigation algorithm that enables each UAV to determine its own movement locally and track pedestrians (mobile targets), it focused on multiple drones to cover a specific area. 
\citeauthor{pen21} took the advantage of DRL to develop an online path planning algorithm based on double deep Q-learning network (DDQN). The constraints were to minimize the energy consumption of the UAV, the objects on the ground were not stationary and were following a Gauss-Markov movement pattern. Author \citeauthor{hua20} aimed to propose a reactive real-time sliding mode control algorithm to navigate a team of UAVs (UAS). The area was divided into multiple sub-areas using the Voronoi partitioning technique, each drone was responsible for a sub-area, he implemented his ideas for both types of tergets, stationary targets and mobile. \\
All the mentioned papers presented their solutions using different simulation software. However, none of them was implemented in the real-world which questions the reliability of the algorithms.



% Part 2

% Part 3



\end{document}

