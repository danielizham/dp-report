\documentclass[../main.tex]{subfiles}

\begin{document}

From traffic monitoring to livestock tracking, and 
military reconnaissance to marine discovery, the \uav
is indispensable.
However, its dependence on a battery as a power supply means
that it has a limited flight time to visit the required
locations. This makes it necessary to minimise 
the mechanical energy by reducing the execution time as much
as possible.
The goal of this project is to develop a system 
that can perform the target visitation task in the shortest
possible time. The objectives are to integrate hardware to make
an autonomous drone, use \gls{ml} 
to detect targets,
and implement \gls{rl} to 
teach the drone how to visit all targets in a minimum amount of time.
The significance of this project is that \uavs will be
able to autonomously capture details of mobile targets with
unknown locations but a particular mobility pattern
while minimising mechanical energy and time.
The hardware part of the proposed solution 
consists of a Raspberry Pi acting as the onboard computer
attached to a Parrot \anafi drone. The onboard computer
also receives high-level commands from the command and control
station. On the software side, \gls{drl} is 
implemented to train a model in the Sphinx 3D
cyber-physical environment. At the same time,
the Olympe program is utilized
to send commands to the simulated and real \anafi drone
and receive sensor data from them. 
In addition, RoboFlow is employed
to build the target detection model using \gls{ml}.
The results so far are an \anafi-Raspberry Pi hardware system
that is able to fly, an \gls{rl} framework that can train the
simulated drone to go to a set position, and a \gls{ml}
model that can identify green cubic targets.
Our research has shown that there is a growing market for
our product because the surveillance landscape
is becoming increasingly dynamic.

\end{document}
