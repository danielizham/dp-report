\documentclass[../main.tex]{subfiles}

\begin{document}

% motivation
% problem
% objectives
% constraints
% solution
% market?
From traffic monitoring to livestock tracking and 
military reconnaissance to marine discovery, the \uav
is indispensable in these applications.
However, its dependant on a battery as a power supply means
that it has a limited flight time to visit the required
locations. This makes it necessary to minimise 
the mechanical energy by reducing the execution time as much
as possible.
The goal of this project is to develop a system 
that can perform the target visitation task in the shortest
time. The objectives are to integrate hardware to make
an autonomous drone, use \gls{ml} 
to detect targets,
and implement \gls{rl} to 
teach the drone how to visit all targets in the minimum time.
The main design constraints have been identified to be 
\SI{15}{minutes} of flying duration and \SI{200}{grams}
of drone payload indicating that the total mass of the 
onboard computer and peripherals must be below this
value. To simplify the process, some of the assumptions
made are that the environment is windless and obstacle-free,
the targets do not move in an arbitrary fashion but follow
an established mobility pattern, and the task is considered
complete when the photos of all targets have been taken.
The hardware part of the proposed solution 
consists of a Raspberry Pi acting as the onboard computer
attached to a Parrot \anafi drone. The onboard computer
also receives high-level commands from the command and control
station. On the software side, \gls{drl} is 
implemented to train a model in the Sphinx 3D
cyber-physical environment. The Olympe program is utilized
to send commands to the simulated and real \anafi drone
and receive sensor data. In addition, RoboFlow is employed
to build the target detection model using \gls{ml}.
Our research has showed that there is a growing market for
our product because the surveillance landscape
is becoming increasingly dynamic.

\end{document}
