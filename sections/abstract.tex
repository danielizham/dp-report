\documentclass[../main.tex]{subfiles}
\graphicspath{{\subfix{../images/}}}

\begin{document}

From traffic monitoring to livestock tracking, and 
military reconnaissance to marine discovery, the \uav
is indispensable.
However, its dependence on a battery as a power supply means
that it has a limited flight time to visit the required
locations. 
Consequently, the visitation needs to be
completed as soon as possible to minimise 
the mechanical energy used by the \gls{uav}.
The goal of this project is to develop a system 
that can perform the target visitation task in the shortest
possible time. The objectives are to integrate hardware to make
an autonomous drone, use \gls{ml} 
to detect targets,
and implement \gls{rl} to 
teach the drone how to visit all targets in a minimum amount of time.
The significance of this project is that 
by combining object detection and \gls{rl}, \uavs will be
able to autonomously capture details of 
fixed targets with unknown locations, 
or mobile targets with an arbitrary mobility pattern,
while minimising mechanical energy and time.
The hardware part of the proposed solution 
consists of a Raspberry Pi acting as the onboard computer
attached to a Parrot \anafi drone. The onboard computer
also receives high-level commands from the command and control
station. On the software side, \gls{drl} is 
implemented to train a model in the Sphinx 3D
cyber-physical environment. At the same time,
the Olympe program is utilized
to send commands to the simulated and real \anafi drone
and receive sensor data from them. 
In addition, RoboFlow is employed
to build the target detection model using \gls{ml}.
The results so far are an \anafi-Raspberry Pi hardware system
that is able to fly, an \gls{rl} framework that can train the
simulated drone to go to a set position, and a \gls{ml}
model that can identify green cubic targets.
Our research has shown that there is a growing market for
our product because the surveillance landscape
is becoming increasingly dynamic.

\vfill
\begin{newrequirements}
    Writing the final report:
    \begin{todolist}
        \item[\done] The abstract cannot be more than \textbf{500 words}.
        \item Go through the Interim report, 
            and update it based on changes that have occurred 
            in your project between last semester and now
            and based on the requirements listed in the 
            Project Guide and the Project Grading Rubrics
        \item Revise the Abstract and enhance it by adding 
            the project’s key achievements and most important 
            conclusions. 
            The last paragraph should highlight the novelty 
            of your design (e.g., what makes your design 
            unique and what are the impacts of your 
            engineered solution, etc.)
        \item Make sure that the tense used is the present 
            tense and the past and not the future 
            (e.g., avoid ‘we will’ or ‘system should’ 
            and report what has been done) 
        \item[\done] \textit{Example of a done item}
    \end{todolist}
\end{newrequirements}
\vspace{0.5cm}

\end{document}
