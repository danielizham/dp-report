\documentclass[../main.tex]{subfiles}
\graphicspath{{\subfix{../images/}}}

\begin{document}

From traffic monitoring to livestock tracking, and 
military reconnaissance to marine discovery, the \uav
is indispensable.
However, its dependence on a battery as a power supply means
that it has a limited flight time to visit the required
locations. 
Consequently, the visitation needs to be
completed as soon as possible to minimise 
the mechanical energy used by the \gls{uav}.
The goal of this project is to develop a system 
that can perform the target visitation task in the shortest
possible time. The objectives are to integrate hardware to make
an autonomous drone, use \gls{ml} 
to detect targets,
and implement \gls{rl} to 
teach the drone how to visit all targets in a minimum amount of time.

The significance of this project is that 
by combining object detection and \gls{rl}, \uavs will be
able to autonomously capture details of 
fixed targets with unknown locations, 
or mobile targets with an arbitrary mobility pattern,
while minimising mechanical energy and time.

The hardware part of the proposed solution 
consists of a Raspberry Pi acting as the onboard computer
attached to a Parrot \anafi drone. The onboard computer
also receives high-level commands from the command and control
station. On the software side, \gls{drl} algorithm called \gls{ppo} is 
implemented to train a model in the Sphinx 3D
cyber-physical environment. At the same time,
the Olympe program is utilized
to send commands to the simulated and real \anafi drone
and receive sensor data from them. 

The \gls{rl} has been proven to be superior in terms of time and
energy spent to execute both fixed-targets and mobile-targets
visitation missions compared to random and zig-zag agents.
Moreover, the work in the visual simulation has been validated in the
real world in the integration testing, thus confirming that it can act
as the digital twin of the physical version.
The framework that has been proposed in this project is flexible
enough to be extended to applications beyond target visitation.

The impact of this project on the global scale is through proving that
\gls{rl} is a viable and effective method to train a \gls{uav} to
execute a visitation mission either with fixed or mobile targets.
Once the framework developed in this project is in place, companies
with little knowledge of \gls{rl} can easily retrain the \gls{uav}
digital twin in the framework for their new missions. 
The simplicity will help spread the use and benefits of \glspl{uav}
more widely, such as in marine discovery, wildlife preservation and
homeland security. 
In addition, this project can lead to the creation of new jobs in
target visitation and area tracking using \glspl{uav}.

\end{document}
