\documentclass[../main.tex]{subfiles}
\graphicspath{{\subfix{../images/}}}

\begin{document}

In the beginning, we set out to achieve six objectives in order to
build an autonomous drone that can accomplish the target visitation
mission in the shortest time and with the least energy.
By the end, we have successfully fulfilled all those objectives, and
a brief summary of how we have done so is laid out
in~\cref{tab:objective-conclusion}.

\begin{center}
    \begin{xltabular}{\textwidth}{ c X X }
        \caption{Our objectives and how we went about fulfilling them.}
        \label{tab:objective-conclusion} \\
        \toprule
        \textit{Objective} 
            & \textit{Description} 
            & \textit{Our approach} \\

        \midrule
        \endfirsthead
        \caption[]{Our objectives and how we went about fulfilling
        them (continued)} \\
        \toprule
        \textit{Objective} 
            & \textit{Description} 
            & \textit{Our approach} \\

        \midrule
        \endhead
        
            1
            & 
            Develop a solution composed of a drone and a command and
            control system which can be used to track fixed and mobile
            targets in the minimum amount of flight time.
            &
            We have used an \anafi drone controlled by the Olympe
            library as a controller that runs on our laptop and acts
            as the command and control system.
            The Olympe controls the \anafi drone to track fixed and
            mobile targets in the shortest time according to a trained
            \gls{rl} model.
            \\ \addlinespace

            2
            & 
            Design and integrate hardware to enable intelligence on
            the drone.
            &
            We have attached and linked a Raspberry Pi to the \anafi
            drone to act as an onboard computer. 
            The Olympe library and \gls{rl} model mainly run in
            the Raspberry Pi instead of the laptop to allow the \anafi
            drone to be fully autonomous whenever the connection with
            the laptop is lost.
            \\ \addlinespace

            3
            & 
            Develop a command and control system that is able to
            provide high-level commands to the drone.
            &
            The command and control system is a laptop that is
            connected to the Raspberry Pi (onboard computer) through
            the \textsc{ssh} or \textsc{vnc} protocols.
            The laptop can control when to start the mission and what
            \gls{rl} model needs to be run on the Raspberry Pi, which
            in turn controls the drone directly.
            \\ \addlinespace

            4
            & 
            Detect and identify targets when they are visited by the
            drone using \gls{ml}.
            &
            Initially, we had trained a custom object detection model
            to detect the targets.
            When it comes to uniquely identifying the targets, we have
            resorted to using \textsc{qr} codes and a pre-trained
            object detection model called Pyzbar to improve on our own
            trained model.
            \\ \addlinespace

            5
            & 
            Minimize the mechanical energy through efficiently
            navigating the drone by leveraging \gls{rl}.
            &
            We have assumed in this project that the mechanical energy
            consumption of the drone is equivalent to its flight time.
            Thus, we have trained an \gls{rl} model using the \gls{ppo}
            algorithm with a timestep reward such that it encourages
            visiting all the targets as fast as possible.
            We have proven the \gls{rl} model's superiority through
            the learning curve during the training in terms of episode
            rewards and episode lengths, and through the comparison
            with the random and zig-zag agents.
            \\ \addlinespace

            6
            & 
            Use cyber-physical simulation in the \gls{rl} training to
            teach the drone how to navigate the targets intelligently,
            and test the navigation in a realistic 3D environment
            before the actual deployment.
            &
            We have utilized the Sphinx simulator to train the \anafi
            drone using \gls{rl}, which is much cheaper and safer than
            doing so in the physical world.
            Also, we have tested the final models in the same
            simulator in which we created a 3D environment including
            the real-world physics that matched the deployment
            settings.
            This proved to be very effective as we were required to
            make only minor modifications before successfully making
            it work as expected during the deployment.
            \\ \addlinespace

        \bottomrule		
    \end{xltabular}
\end{center}

The strengths of our solution are attributable to our use of \gls{rl} in training
the agent.
The \gls{rl} allows us to execute a target-visitation mission without
knowing the targets' locations beforehand as long as the targets'
initial positions and movements follow certain distributions and are
not random. 
It is versatile in a way that any mobility pattern of the targets can
be theoretically learned by the agent after the agent has been retrained.

On the other hand, our main shortcomings stem from the use of
a simulator and programs that are not fully open-source.
As a result, many processes such as creating the 3D environment,
controlling the drone and applying the \gls{rl} algorithm were limited
because we could not modify the underlying code.
Another shortcoming is the need for long training hours to get the
agent intelligent enough through \gls{rl}.
Personal laptops are sufficient only for simple applications like
ours.
There are other shortcomings which we have outlined as challenges in
\cref{sec:future}.

The key contributions of our solution lie in the fact that the framework
is flexible to cater to target visitation applications beyond what
this project entails.
The steps remain the same, but things such as the targets' mobility
pattern, reward function and \gls{rl} algorithm may need to be
changed.
Moreover, the framework provides a digital twin of the real
environment in the form of visual simulation to facilitate the
exploration of smart navigation.

\end{document}
