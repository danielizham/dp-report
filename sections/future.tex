\documentclass[../main.tex]{subfiles}
\graphicspath{{\subfix{../images/}}}

\begin{document}

\subsection{Hardware}

Many improvements can be added to the hardware section in our future work,
such as upgrading the Raspberry Pi memory version to 8\,\textsc{gb}.
This is because in future improvements, adding more complex models,
mobility patterns, and features will consume more memory and
4\,\textsc{gb} will not be sufficient for that.
Also, the Raspberry Pi battery should be replaced since after so many
tests and full charges, it has degraded and its life cycle has
decreased significantly. 

Another improvement needed is to use the new Parrot \anafi Ai which comes
with many professional features and enhancements such as longer
battery life, more secure protocols, more competent navigation, and
the 4G-connection support which is essential in our autonomous
application.
This feature will make the connection between our command and control
system and the drone very convenient without the need for the on-board
computer. 
However, the price of the drone is high and it is not widely
available to buy.

\subsection{\textsc{Rl} and Simulation}

\Cref{tab:shortcomings} details the shortcomings we have identified in
our project while \cref{tab:additional-features} describes additional
features we could implement as future work with regards to \gls{rl}
and simulation portions of the project.

\begin{center}
    \begin{xltabular}{\textwidth}{ X X }
        \caption{Shortcomings and our proposed solutions in simulation
        and \gls{rl} parts of the project.}
        \label{tab:shortcomings} \\
        \toprule
        \textit{Shortcoming} 
            & \textit{Proposed solution} \\

        \midrule
        \endfirsthead
        \caption[]{Shortcomings and our proposed solutions in simulation
        and \gls{rl} parts of the project (continued)}\\
        \toprule
        \textit{Shortcoming} 
            & \textit{Proposed solution} \\

        \midrule
        \endhead
        
        Unable to use the \gls{ppo} workers in the training            
        & 
        Sphinx allows for spawning of multiple drones but they all
        need to be in the same environment. 
        \gls{ppo} workers ought to be in their own separate
        environments.
        To overcome this, we may choose an open-source drone
        which will allow us to use the vanilla Gazebo to simulate it.
        We are then able to create different environments for
        different simulated drones thus qualifying them to be
        \gls{ppo} workers.
        \\ \addlinespace

        Hardcoding the boundary of the agent
        &
        Instead of making the drone hover when it decides to go
        outside the grid, the agent could be made more autonomous by
        letting it learn in the training that going outside the
        boundary will incur a big negative reward.
        \\ \addlinespace

        No energy in the reward equation
        &
        The RL may incorporate the energy used for each action in the
        reward. This is useful for the time when the drone must decide
        between two or more actions that lead to the same destination,
        but one of them obviously consumes the least energy.
        Also, the energy calculation of the \gls{rl} agent will be
        lower compared to other baseline agents.
        \\ \addlinespace

        Non-optimal hovering movement of the agent
        &
        The agent's hovering movement lasts instantaneously when it
        should have few-second delay.
        The delay was not implemented because the training runs twice
        the speed as normal but the delay would stay the same as it
        would be defined using the wall clock time.
        As a result, the agent will learn a wrong hovering movement
        than what it is supposed to be in the real world.
        As a proposed solution, we could query Sphinx for the
        real-time/simulation-time ratio and define the delay with
        respect to this ratio instead of the wall clock time.
        \\ \addlinespace

        \bottomrule		
    \end{xltabular}
\end{center}

\begin{center}
    \begin{xltabular}{\textwidth}{ X X }
        \caption{Additional features and their descriptions in
        simulation and \gls{rl} parts of the project.}
        \label{tab:additional-features} \\
        \toprule
        \textit{Additional feature} 
            & \textit{Description} \\

        \midrule
        \endfirsthead
        \caption[]{Additional features and their descriptions in
        simulation and \gls{rl} parts of the project (continued)}\\
        \toprule
        \textit{Additional feature} 
            & \textit{Description} \\

        \midrule
        \endhead
        
        Targets with more advanced mobility patterns
            & 
        The project's applications can be extended to cover other
        mobility patterns such as Manhattan and reference point group
        mobility (\textsc{rpgm}).
        The targets will need to be programmed using their plugins to
        behave as the chosen mobility pattern and then the drone can
        be trained as we did in this project.
            \\ \addlinespace

        Train the agent in the cloud
            &
        A cloud provider, such as \textsc{aws}, Azure and Google Cloud
        Platform, offers a \textsc{vm} that can have a much more
        powerful \textsc{gpu} compared to consumer laptops.
        This will facilitate the training immensely especially when
        the targets' mobility pattern is complex.
            \\ \addlinespace

        \bottomrule		
    \end{xltabular}
\end{center}

\end{document}
