\documentclass[../main.tex]{subfiles}
\graphicspath{{\subfix{../images/}}}

\begin{document}

\subsection{Abdelrahman Soliman}

I learned many lessons from this project such as appreciating 
the time and never postponing any task, this will lead to smooth 
the progress of any project and will make my the tasks look 
easy and simple. Another lesson learned is you learn from 
your failures, I heard this lesson in many quotations but I 
didn't believe in it until I faced that failure in doing some of 
my tasks and when I face a similar problem, I solve it 
immediately and with confidence.

For the professional skills, I gained experience in the hardware 
section and microcontrollers which boosted my confidence which 
was low before the project, this experience will help me in 
future research and the projects that I am planning to do after 
I graduate. 
Web development skill was one of my interests, 
unfortunately, as a computer engineer, I didn't take any web 
development course but this project gave me the chance to learn 
and apply the basics blocks that am aiming to improve 
and advance in the future. 
Make use of programs and services that simplify
your work, such as GitHub which have been used to share the 
Latex files and the code needed to write this report.
Another critical skill that I gained 
are the basics of writing a scientific paper and I want to 
appreciate Dr.Amr who illustrated how to divide the sections of 
the paper and the main mistakes that people fall into so that 
we can avoid them.

About the technical skills, I learned from the multiple meetings how to 
approach the problem from different sides and how to find the 
the best solution for it. Also, I learned to be more responsible
and professional by finally getting rid of one of my bad behavior 
which is procrastination.

Avoid getting distracted from the main goal or objective in 
future projects, in the first semester I wasted 2-3 weeks 
on tasks that I thought it's important to the project but 
I found that they are not so analyzing the tasks and 
order their importance is an important thing before
you start any one of them.

Teamwork experience is a nice lesson to learn and 
there are different kinds of team members and team roles.
I learned to fill the gaps and how to fit myself in different 
positions to boost the progress of the project. 
Finally, communication skills are very crucial to make smooth 
the flow of the work and to deliver what is required accurately.

\subsection{Mohamad Mohamad Ali Bahri}
This senior project has added a lot to my experience and knowledge, I learned
various new skills and enhanced my knowledge in numerous existing ones. 
Aside from the technical skills, I learned many personal and soft skills that 
could really play a huge role in my future. 

For social and personal skills, I have profited from my teammates great 
communication skills and excellent critical thinking. They were very helpful, 
kind, and supportive throughout the year. A major thing that I had to put away
and leave was my procrastination as I understood how time sensitive things can
be. 

For technical skills most things we worked on were not things that we have 
studied in the university which made this project a lot more challenging for 
us but more beneficial at the same time. 

One of the many things I learned from this project is the result of good 
preparation as the project's success was dependent on our preparation over the 
college years in general, and the preceding summer in particular as we spent 
the summer studying and helping each other to make this project work.

All in all, this year and this project was something memorable that I will not 
forget in my entire life. Bitter or sweet, it was an amazing experience that 
every engineering student has to go through in order to excel, learn and 
graduate.

\subsection{Mohamed Daniel Bin Mohamed Izham}

% Personal skills
This project has taught me the importance of time organization and
having clear goals in mind.
Without them, the objectives will not be met in time caused by mainly
doing things that do not contribute to the project.
Two valuable and related skills that I have gained are problem solving
and resourcefulness largely because things invariably do not go as
planned and bugs are part and parcel of the programming life. 
To solve them, most of the times, there is no straightforward way but
we needed to come up with an ingenious hack.
Some other times, we needed to realise that there is no fix given our
limited time so we had to resort to an alternative library or program.
One less appreciated skill is the ability to apply existing knowledge. 
I have learned to identify the tasks that can make use of my
existing knowledge to save time.
This follows that we should not forget the important concepts from
previous courses as many of them are about the fundamentals, which are
costly for a project if we get them wrong.
In addition, the previous courses have allowed me to practise good
study techniques which I have used to learn new concepts for this
project.

% Interpersonal skills
By working in a team, I have learned to communicate and collaborate
effectively.
Integral in these is the ability to understand the situation of your
team members.
Only then we can begin to delegate tasks according to each other's
expertise which is another interpersonal skill that is highly
sought-after.
Other common skills I have gained are the ability to share
constructive ideas, to adapt with the team when there are changes
required and to act professionally such as when communicating with the
\textsc{rc} Club, the \gls{ieee} and Computer Science and Engineering
Department.

% Technical skills
In terms of the technical skills, in setting up the simulation and
\gls{rl}, I have learned advanced \gls{oop} and design patterns.  
Along with that are the \gls{rl} concepts which I have found to be
fascinating -- I have always been amazed how the pioneers in this
field were able to think up these concepts in the first place.
I have also learned to use \LaTeX to create documents.
The learning curve is certainly steeper than the Microsoft Word, but
its consistency and flexibility make it worth the time.
Another important tool that I have learned to master is GitHub.
It was integral in our collaboration workflow and we could not imagine
how much longer our work would have taken had we not used it.

% shortcomings
Nevertheless, there were some important shortcomings which we would
want to avoid or improve in the future.
The first thing that we would do differently is to carry out a
preliminary integration testing earlier so that we would have more time
to identify the errors and correct them.
Another improvement is to use Trello to organize our work based on
Kanban.
We were communicating our tasks in-progress and pending ones by
WhatsApp and Teams which I found to be cumbersome. 
A centralised place like Trello would definitely allow us to save more
time in that department.

% in my future career
All the skills I have mentioned may be transferable in my career in the
future.
While the programming language might be different, GitHub will certainly
be used for version control.
I however regret to realise that most probably, the job that I will
get will not involve my hard-earned \gls{rl} knowledge as it is only
demanded in a special niche market or research.

\end{document}
