\documentclass[../main.tex]{subfiles}

\begin{document}

% {{{ Motivation
The \uav is one of the most promising devices in the 
21\textsuperscript{st} century. Commonly known as drone, 
it improves many existing applications and makes possible 
innumerable new ones. It can be used to monitor livestock, 
carry out security surveillance, manage traffic 
and discover wildlife.
In addition, it can facilitate search and rescue missions,
aid in military operations and expedite final mile delivery.
This is because the \uav combines the best of both worlds
of \textsc{cctv}s and satellites. 
While a fixed \textsc{cctv} is good at taking 
high resolution images and a far away satellite can monitor
large areas, a \uav can do both equally well without 
being restricted in location nor image quality. 
In marine biology for instance,  
the study of interactions between members of a migrating 
pod of whales can be done using a \uav but not 
\textsc{cctv}s nor satellites.
However, micro-drones, which is the most common type of \uavs,
suffer from limited flight time. This is especially 
disadvantageous in
the mentioned tasks because they require visitation of targets
whose pattern of movement is the only information known.
% }}}

\subsection{Problem statement}

% {{{ Problem statement
More often than not, individuals and organizations face the problem
of visiting a set of destinations in the fastest time 
possible.
For example, an operation engineer needs to monitor
a number of reactors and plants within his or her shift.
The need to do so in the shortest time can be due to restrictions on
cost, energy or other limited resources of concern.
This problem is commonly known as the travelling salesman problem
where historically, a salesman had to make a decision in which order
to visit a set of cities exactly once to conduct his trade
in the most time-saving manner.
Depending on the metric of interest, 
the destinations have to be visited in the correct order.
Algorithms, such as Dijkstra's, have been successfully used 
to find the shortest path between the cities.
However, in applications where the \uav with a limited
flight time is used to 
visit targets which are mobile, the problem is intensified.
As illustrated in \cref{fig:problem}, this
is what the project at hand intends to solve.
Given the mobile targets' mobility pattern,
how do we make a \uav able to visit them
in the fastest possible time and in turn, 
the most energy-saving way.  
% }}}

% {{{ Figure of the problem
\begin{figure}[tb] 
    \centering
    \includegraphics[width=0.9\textwidth]{problem.png} 
    \caption{The \gls{uav} needs to calculate 
    the best trajectory to cover the red cars
    as fast as possible before it runs out of battery.} 
    \label{fig:problem} 
\end{figure}
% }}}

% {{{ Technical and non-technical challenges
There are many inherent challenges that solving this problem
presents. First of all, it is difficult to make a drone
fly autonomously to visit multiple targets whose locations
are not known. 
This is a common scenario in various applications. 
Moreover, this needs to be done in the minimum
amount of flight time.
Otherwise, several \uavs need to be used concurrently
or sequentially.
Secondly, flying a drone efficiently is complicated 
because it has six degrees of freedom. In other words, 
there are many parameters that the controller needs to 
take into account to move 
it precisely to a destination point. 
Thirdly, there will be obstacles that the autonomous \uav
needs to know how to avoid while visiting the targets.
Finally, the fact that \uavs generally use a battery
as a power supply makes it imperative to save energy 
while flying it.
Based on the goal of the project, the challenges that will be 
tackled are the autonomous navigation
of the drone to visit identifiable targets in the shortest
time and energy-saving way.
% }}}

% {{{ Solution
From a high level, the solution will involve machine learning
to uniquely identify the targets, and reinforcement learning
to make the drone autonomously visit the targets in the
shortest time possible. In addition, there will also be
a hardware integration between a drone and an onboard
computer. This is to avoid the command and control
station sending the commands to the drone. Instead, the drone
will possess its own computer that it will use to 
autonomously visit the targets. Consequently, the system 
will also be scalable.
% }}}

\subsection{Project significance}

% {{{ Project importance
The significance of the project is enormous. 
Making a \uav able to intelligently 
visit mobile targets in the shortest time
and minimal mechanical energy 
opens up doors to many existing and new applications
due to the \uav's unique features.
First of all, the \gls{uav} combines the best features
from \textsc{cctv}s and satellites.
Using \glspl{uav} gives the benefit of high resolution images
for the purposes of surveillance and monitoring
similar to \textsc{cctv} cameras 
without the restriction in movement~\cite{Sha19}.
At the same time, it allows for large-scale monitoring 
usually done exclusively using satellites 
without its low accuracy~\cite{Sha19}.
This feature of the \uav is useful in applications
such as marine biology, livestock monitoring and military
reconnaissance.
Second of all, target visitation done by a \gls{uav}
is faster because it has a higher degree of freedom
compared to terrestrial vehicles, and
it flies at a higher altitude 
making it more able to avoid 
many of the surrounding obstacles.
This allows it to monitor traffic and make last mile delivery
among others.
% }}}

% {{{{ Describe how can stakeholders use the project,
% ## its uses/applications,
% ## impacts if the design problem was not solved,
% ## and benefits of the project in Qatar and the region.
Based on the project's significance, we have identified 
its stakeholders to be 
those companies, institutions and organizations which
have certain fixed or mobile subjects of interest 
that they want to monitor, capture or track.
By using our product,
they will be able to make the aforementioned applications
faster or,
better yet, possible.
On the other hand, the absence of this project will mean that
they will remain
time-consuming, tedious and expensive
while some will remain infeasible
causing the stakeholders to be deprived of ample growth opportunities.
In some instances, the difference between using 
and not using an intelligent \gls{uav} 
\gls{drl}-trained on mobile target visitation
will be a matter of life-and-death
such as in the search-and-rescue field.

Moreover, Qatar stands to benefit the most from this project 
considering the \textsc{fifa} 2022 it will be hosting
and its rapidly transforming economy and population.
For example, the use of drones can help manage 
the \textsc{fifa} World Cup 
and the increase in population through better monitoring 
of traffic and people, whose data can contribute
to better studies in those sectors.
In addition, the Covid 19 has made online shopping
the number one choice for 
the majority of people to buy goods in Qatar~\cite{Has20}.
The success of this project can help in delivering
the goods to intended destinations with minimum delay,
thus contributing to a healthy economy.
% }}}

% {{{ Why do you want to do this project? 
% # What got you interested in it? 
% # And how this project will help you further your career goals?
The project significance even extends to the authors
on a personal and individual level.
The prime evidence that has attracted the authors to invest
their energy and time in this project is
the increasing number of giant companies like Amazon, Alphabet
and Microsoft experimenting with and profiting from \glspl{uav}
to improve their operations~\cite{Jun17}.
Target visitation is an important subtask of these among others.
Therefore, we believe there is a huge value
in this project and upon the successful completion of it,
there will be many doors opening up to us in terms of
career opportunities because 
this technology is in high demand yet, 
people with the right competency
are scarce.
Besides the jobs that are directly related to the project,
there are other fields that will appreciate the knowledge
and skills that we will gain from the methodological processes 
of designing the simulation,
training the \gls{drl} models and testing them in the real world.
These include robotics, machine learning, and 
network security.
% }}}

% # Possibly cite relevant references that discuss 
% # the importance of our project

\subsection{Project objectives}\label{sec:objectives}

% {{{ Express clear specific objectives
The aim of this project is to create a process 
by which a cyber-physical simulator is used to train 
a \gls{drl} model. This model will then be used
to make a \gls{uav} accurately accomplish a
mobile target visitation task.
The project's detailed objectives are as follows:

\begin{enumerate}
    \item \label{obj:overview}
        Develop a solution composed of a drone and
        a command and control system which can be used
        to track fixed and mobile targets in the
        minimum amount of flight time.
    \item \label{obj:hardware} Design and integrate 
        hardware to enable 
        intelligence on the drone.
    \item Develop a command and control system that
        is able to provide high-level commands 
        to the drone.
    \item \label{obj:machine-learning} 
        Use \gls{ml} to detect and identify
        targets and their locations.
    \item \label{obj:drl} Use \gls{rl}
        to intelligently
        navigate the drone to try to visit all the targets
        in the shortest time possible.
    \item \label{obj:simulation} 
        Use cyber-physical simulation in the \gls{rl} training
        to teach the drone how to navigate the targets
        intelligently, and test the navigation in a realistic
        3D environment before the actual deployment.
\end{enumerate}
% }}}

% {{{ Deliverables and desired results
To demonstrate the objectives, there will be a number 
of deliverables that we will produce.
Firstly, we will present a detailed report 
that will elaborate on our aim, objectives, 
literature review, methodology,
solution and the market value of the project.
Secondly, we will output a trained \gls{drl} model for 
the target visitation task based on a specific
mobility pattern.
Finally, we will build a \gls{uav} system
consisting of the \anafi drone as well as a Raspberry Pi
that will contain the model making the
\anafi drone fully autonomous.
Based on the deliverables, the desired result is that
given the same mobility pattern for 
the moving targets as the one in the \gls{drl} training,
the \gls{uav} should be able to fly to locations it has learned
that allow it to capture the maximum targets. 
% }}}

\end{document}
