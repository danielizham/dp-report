\documentclass[../main.tex]{subfiles}

\begin{document}

\subsection{Project milestones}

\begin{table}[H]
    \centering
    \caption{Senior 1 project milestones.}
    \label{tab:milestones}
    \begin{tabularx}{\textwidth}{ X p{5cm} p{9cm} }
        \toprule

        \textit{No.} & \textit{Milestone} 
            & \textit{Description}
        \\

        \midrule

        1 & Simulation environment
            & Setup the environment and be familiar with the software and libraries that will be used.
        \\
        2 & Object detection model
            & Create dataset and start training the model then try to enhance it.
        \\
        3 & Research hardware interfacing and possible solutions
            & Collect information about possible hardware components and architectures to be used.
        \\
        4 & Literature review and related work
            & Start reading related work and previous papers related to our topic, compare, criticize, and summarize them.
        \\
        5 & Hardware architecture and implementation
            & Start grouping the components together and see what is missing and from where to get it.
        \\
        6 & Enhance the simulation environment
            & Apply the detection model and tweak the environment and improve it.
        \\
        7 & Hardware testing and proving design constraints parameters
            & Test the design constraints and push the drone and hardware components to the limit.
        \\
        8 & Interim report
            & Finalize the different sections of the report
            and compile them.
        \\
        9 & Report review and corrections
            & Review the written report with the supervisor
            and make corrections as necessary.
        \\
        10 & Presentation
            & Create the presentation based on the report.
        \\

        \bottomrule
    \end{tabularx}
\end{table}

\subsection{Project timeline}

% Hardware timeline
\begin{center}
	\setcounter{milestonecounter}{0}
	\def\mysection{hardware} % <- change this variable
	\begin{small}
		\begin{xltabular}{\textwidth}{ p{3cm} p{5cm} X X c }
			\caption{Senior 2 project timeline for \mysection.}
			\label{tab:timeline-hardware} \\ % <- change this label
			
			\toprule
			\textit{Task} 
			& \textit{Description} 
			& \textit{Start Date} 
			& \textit{End Date} 
			& \textit{Assigned To} \\
			
			\midrule
			\endfirsthead
			\caption[]{Senior 2 project timeline
                        \mysection\ (continued)}\\
			\toprule
			\textit{Task} 
			& \textit{Description} 
			& \textit{Start Date} 
			& \textit{End Date} 
			& \textit{Assigned To} \\
			
			\midrule
			\endhead
			
			\addlinespace
			\multicolumn{2}{p{8cm}}{\raggedright
				\emph{%
					Milestone \showmilestonecounter:
					Hardware testing and proving
					design constraints parameters
				}
			}
			& \emph{24-01-2022} & \emph{10-02-2022} & 
			\\ \addlinespace
			
			Task \thesubcounter: 
			Test command and control
			maximum coverage range
			& Place the drone further and further away and see when it fails to
			respond to user interface actions or view video stream.
			& & & Abdulrahman \\
			
			Task \thesubcounter: 
			Test flying duration \& batteries
			& Conduct time comparison experiments for both drone's
			battery (2700mAh) and the Raspberry Pi battery (4000mAh) 
			using different actions
			& & & Abdulrahman \\
			
			\addlinespace
			\multicolumn{2}{p{8cm}}{\raggedright
				\emph{%
					Milestone \showmilestonecounter:
					3D print design
				}
			}
			& \emph{10-02-2022} & \emph{20-02-2022} & 
			\\ \addlinespace
			
			Task \thesubcounter: 
			List some possible 3D designs
			& Search for simple and efficient 3D 
			designs for holding the on-board computer
			above the drone.
			& & & Abdulrahman \\
			
			Task \thesubcounter: 
			Collect the dimensions
			&  measure dimensions of the on-board computer 
			and the free drone surface. 
			measure the possible size of the 
			3D parts.
			& & & Abdulrahman \\
			
			Task \thesubcounter: 
			Print the design
			& Print the parts and choose the best material type and
			fill rate.
			& & & Abdulrahman \\
			
			Task \thesubcounter: 
			Adjust the dimensions and Reprint
			& Some parts needed some dimensions change
			so remeasure and reprint the parts again. 
			& & & Abdulrahman \\
			
			Task \thesubcounter: 
			Attach 3D printed parts to the drone
			& stick the parts together using superglue
			then use zip tie to hold it with drone. 
			& & & Abdulrahman \\                
			
			\addlinespace
			\multicolumn{2}{p{8cm}}{\raggedright
				\emph{%
					Milestone \showmilestonecounter:
					Moving target solution
				}
			}
			& \emph{20-02-2022} & \emph{28-02-2022} & 
			\\ \addlinespace
			
			Task \thesubcounter: 
			List some possible moving targets
			& Discuss the possible solutions such
			as RC cars or Robots
			& & & Bahri \\
			
			Task \thesubcounter: 
			Borrow \& Build the Ev3 robots
			& Get the Ev3 robots from the department
			and start building them
			& & & Bahri \\
			
			Task \thesubcounter: 
			Program the Ev3 robots
			& Code the robots to move in
			specific probability in specific direction
			such as N,NE,NW,S,SE,SW
			& & & Bahri \\
			
			\bottomrule
		\end{xltabular}
	\end{small}
\end{center}

% RL timeline
\begin{center}
    \setcounter{milestonecounter}{0}
    \def\mysection{\textsc{rl}} % <- change this variable
    \begin{small}
        \begin{xltabular}{\textwidth}{ p{3cm} p{5cm} X X c }
            \caption{Senior 2 project timeline for \mysection.}
            \label{tab:timeline-rl} \\ % <- change this label

            \toprule
            \textit{Task} 
                & \textit{Description} 
                    & \textit{Start Date} 
            & \textit{End Date} 
                & \textit{Assigned To} \\

            \midrule
            \endfirsthead
            \caption[]{Senior 2 project timeline \mysection\ (continued)}\\
            \toprule
            \textit{Task} 
                & \textit{Description} 
                    & \textit{Start Date} 
            & \textit{End Date} 
                & \textit{Assigned To} \\

            \midrule
            \endhead

            \addlinespace
            \multicolumn{2}{p{8cm}}{\raggedright
                \emph{%
                    Milestone \showmilestonecounter:
                    Target simulation
                }
                }
                & \emph{24-01-2022} & \emph{10-02-2022} & 
            \\ \addlinespace

            Task \thesubcounter: 
            Code the target mobility pattern.
                & Give each target the ability to move in 8 directions 
                -- 70\% of the time north-west the rest randomly among
                the remaining directions.
                & & & Daniel \\

            Task \thesubcounter: 
            Change the appearance of the targets
                & Cover the targets with a unique QR code so that they
                can be individually identified during the training and
                actual mission
                & & & Daniel \\

            \addlinespace
            \multicolumn{2}{p{8cm}}{\raggedright
                \emph{%
                    Milestone \showmilestonecounter:
                    New object detection model
                }
                }
                & \emph{01-02-2022} & \emph{10-02-2022} & 
            \\ \addlinespace

            Task \thesubcounter: 
            Integrate OpenCV QR code detection
                & Try to use the OpenCV QR code detection library in
                the training to uniquely identify the targets.
                & & & Daniel \\

            Task \thesubcounter: 
            Integrate Pyzbar QR code detection
                & The OpenCV QR code library did not allow us to place
                a logo in the middle of the QR code which is common in
                its latest version.
                The logo will contain the ID number which ease
                debugging for us as it is not possible for humans to
                identify the ID by just looking at the code.
                Pyzbar allows the logo in the centre, hence it is
                chosen instead of OpenCV.
                & & & Daniel \\

            \addlinespace
            \multicolumn{2}{p{8cm}}{\raggedright
                \emph{%
                    Milestone \showmilestonecounter:
                    RL agent training for fixed targets mission
                }
                }
                & \emph{10-02-2022} & \emph{20-02-2022} & 
            \\ \addlinespace

            Task \thesubcounter: 
            Generate a distribution
                & Create a multivariate distribution representing
                the probability of being in a location in x-y
                coordinates.
                & & & Daniel \\

            Task \thesubcounter: 
            Make the targets listen to the changing locations
                & Add in the target's plugin the ability to listen to
                changes in a csv file and extract the information
                intended for the current target only based on its name
                & & & Daniel \\

            Task \thesubcounter: 
            Create a PPO agent
                & Make the \anafi drone as a PPO agent so that it can
                be trained
                & & & Daniel \\

            Task \thesubcounter: 
            Train the agent
                & Run the agent for 50,000 timesteps through
                trial-and-error until it learns and improves based on
                the returns
                & & & Daniel \\ 

            \addlinespace
            \multicolumn{2}{p{8cm}}{\raggedright
                \emph{%
                    Milestone \showmilestonecounter:
                    RL performance verification for fixed targets mission
                }
                }
                & \emph{20-02-2022} & \emph{28-02-2022} & 
            \\ \addlinespace

            Task \thesubcounter: 
            Code the agent for testing
                & Allow the agent to use the learned model in
                completing the mission 
                & & & Daniel \\

            Task \thesubcounter: 
            Code the zig-zag pattern agent
                & Define the zig-zag path that the \anafi drone needs 
                to take in the simulation to act as a baseline
                & & & Daniel \\

            Task \thesubcounter: 
            Code the random pattern agent
                & Define another agent that moves in the 8 directions
                uniformly to act as a baseline
                & & & Daniel \\

            Task \thesubcounter: 
            Run all agents
                & Test all agents by running them in the simulation
                for 10 episodes and the fixed targets change locations
                based on the same distribution as in the \gls{rl} training
                & & & Daniel \\

            Task \thesubcounter: 
            Calculate and compare
                & Based on the chosen movements, calculate the speed
                and energy and then compare them to see which agent
                performed the best 
                & & & Daniel \\ 
                
            \addlinespace
            \multicolumn{2}{p{8cm}}{\raggedright
                \emph{%
                    Milestone \showmilestonecounter:
                    RL agent training for mobile targets mission
                }
                }
                & \emph{13-03-2022} & \emph{30-03-2022} & 
            \\ \addlinespace

            Task \thesubcounter: 
            Code 3 agents' movements
                & Use the previously developed movement pattern for
                only 3 targets 
                & & & Daniel \\

            Task \thesubcounter: 
            Code 7 agents' locations
                & Generate a different distribution for the fixed
                positions of 7 of the targets 
                & & & Daniel \\
  
            Task \thesubcounter: 
            Make the targets listen to the changing locations
                & Add in the target's plugin the ability to listen to
                changes in a csv file and extract the information
                intended for the current target only based on its name
                & & & Daniel \\

            Task \thesubcounter: 
            Create a PPO agent
                & Make the \anafi drone as a PPO agent so that it can
                be trained
                & & & Daniel \\

            Task \thesubcounter: 
            Train the agent
                & Run the agent for 150,000 timesteps through
                trial-and-error until it learns and improves based on
                the returns
                & & & Daniel \\ 

            \addlinespace
            \multicolumn{2}{p{8cm}}{\raggedright
                \emph{%
                    Milestone \showmilestonecounter:
                    Log and video saving feature
                }
                }
                & \emph{27-03-2022} & \emph{02-04-2022} & 
            \\ \addlinespace

            Task \thesubcounter: 
            Add logging feature
                & Use Python built-in logging module to log the time,
                current \gls{gps} location and the next action of the
                drone.
                & & & Daniel \\

            Task \thesubcounter: 
            Save camera feed
                & Use Olympe to capture and save the frames coming
                from the drone's camera in a video file.
                & & & Daniel \\

            \addlinespace
            \multicolumn{2}{p{8cm}}{\raggedright
                \emph{%
                    Milestone \showmilestonecounter:
                    Prepare the environment for full integration testing
                }
                }
                & \emph{03-04-2022} & \emph{21-04-2022} & 
            \\ \addlinespace

            Task \thesubcounter: 
            Change the environment
                & The cell size, targets' size and movements and many
                other parameters need to be adjusted and verified that
                the agent could still work using the trained
                fixed-targets and mobile-targets models
                & & & Daniel \\

            \bottomrule
        \end{xltabular}
    \end{small}
\end{center}

% UI timeline



\subsection{Anticipated risks}

\begin{table}[H]
	\centering
	\caption{Anticipated Risks.}
	\label{tab:Anticipated Risks}
        \begin{tabularx}{\textwidth}{ X X X }
		\toprule
		\textit{Anticipated Risk} 
		& \textit{Description} 
		& \textit{Minimizing approach} \\
		
		\midrule
		
		Late delivery of the components 
		& Some components available online 
		and it takes time to arrive, 
		any delay will affect the hardware implementation.
		& Order early and from trusted sources with previous 
		customers reviews. \\
		
		Lose connection to the Raspberry Pi
		& Because we are connecting using Wi-Fi, maybe a connection loss 
		happen to due coverage or hardware issue.
		& Code a script to return the drone to the takeoff location 
		and do multiple tests to make sure we cover all the possible cases. \\
		
		Lose control of the drone
		& Because we are limited in fly time and battery capacity, 
		the drone can shut down immediately if no power reaches it.
		Another reason can be that software bugs and crashes can 
		result in unexpected behavior like flying in random directions.
		& For the fly time, we will make sure we land the drone before it reaches the flight time constraints.
		We will create a handlers script for the software crashes 
		to ensure that Raspberry Pi gets the control again.  \\
		
		Flying drone outdoor without permissions 
		& Qatar has very restricted rules regarding flying a drone 
		with a camera outdoor.
		& Get the required permissions to fly the drone in 
		Al-Khor airport or fly it indoors inside the university with permission from the department. \\

                Mismatch between the development and 
                simulation environments
                & The development platforms for Sphinx, Olympe
                and \gym may not use the same \textsc{os}
                and libraries as running them. For example,
                developing plugins for Sphinx has to be done
                on Ubuntu 16.04 only, but running it can be
                done on Ubuntu 18.04.
                & Use Docker to run a suitable container 
                for the development platforms.
                \\

                Accidental premature termination of the 
                training
                & The \gls{rl} training can last for a few
                days. Many things can happen during that
                period such as forgetting to plug in the
                power supply, disconnection between 
                Olympe and Sphinx, and sudden shutdown of
                the laptop due to overheating.
                & Save checkpoints of the weights and
                the last learning rate from time to time
                so that the training can be resumed from
                the point before the termination.
                \\
		
		\bottomrule		
	\end{tabularx}
\end{table}

\end{document}
