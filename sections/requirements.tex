\documentclass[../main.tex]{subfiles}

\begin{document}

\subsection{Functional requirements}

\blindtext
\newpage
\subsection{Design constraints}

\begin{table}[hbt!]
    \centering
    \caption{Technical design constraints}
    \label{tab:technical-design-constraints}
    \begin{tabular}{ P{2.8cm} P{12.2cm} }
        \toprule
        \textit{Name} 
            & \textit{Description} \\

        \midrule

        Power supply  
            & The Anafi and the Raspberry Pi must be 
            battery powered because they are mobile 
            (with ?? voltages DC/mAh)  \\

        Flying range/altitude 
            & The flying range of the Anafi is limited 
            by the signal strength of the communication 
            between the computer and the Raspberry Pi. 
            Also, the altitude is dictated by the Anafi’s 
            maximum height it can reach and the performance 
            of the drone at a certain height. \\

        Flying time  
            & The maximum flying time must be lower than 
            the Anafi’s 25 minutes max flight lower than 
            the Raspberry Pi’s ?? minutes \\ 

        Transmission delay  
            & The delay in the communication between 
            the Raspberry Pi and the Anafi must be acceptable \\

        Payload/weights  
            & The Raspberry Pi with its peripherals should not 
            exceed the maximum load the Anafi can carry. \\

        \bottomrule
    \end{tabular}
\end{table}

\begin{table}[hbt!]
	\begin{tabular}{| P{2.4cm} | P{2cm} | P{10.5cm} |}
		\hline 
		\textit{Type} & \textit{Name } & \textit{Description} \\ \hline
		
		Economic & Cost & The cost of the entire system must not exceed x \$ \\ \hline 
		
		Environmental & Light source & The Anafi requires good lighting to take pictures of the targets \\ \hline
		
		Safety/Security & Secure communication & The communication between Anafi-Raspberry Pi and Raspberry Pi-computer should be uninterrupted and secured against interference \\ \hline
		
		Safety/Security & Safety & The Anafi must know how to return to base should there be a discommunication with the computer \\ \hline
		
		Ethical & Privacy & The drone must not invade the privacy of the entities other than the targets \\ \hline
		
		Ethical & Privacy & The monitored targets must be acceptable by law to be tracked \\ \hline
		
		Environmental & Eco-friendly & The system only uses electricity and emissions \\ \hline
		
		Sustainability & Modularity & The system must be such that the DRL algorithm can be swapped with an improved one easily \\ \hline 
		
		Reliability & Efficiency & The DRL should perform as expected all the time \\ \hline
		
	\end{tabular}
	\caption{Practical design constraints.}
\end{table}\label{tab: practical-design-constraints}

\newpage
\subsection{Design standards}

\begin{table}[hbt!]
	\begin{tabular}{| P{2.2cm} | P{12.3cm} |}
		\hline 
		\textit{Standard} & \textit{Usage}\\ \hline
		IEEE 802.11 & To be used in the communication between the Raspberry Pi and the Anafi and the Raspberry Pi and the computer \\ \hline 
		
		WPA2 & To be used in securing the communications above \\ \hline
		
		GPS & To be used by the Anafi to convey its position to the Raspberry Pi \\ \hline 
		
		SSH & To be used between the Raspberry Pi and the computer \\ \hline
		
		JSON-RPC version 2.0 & Communication protocol used to control the components of the simulated Anafi \\ \hline 
		
		Google Protocol Buffers & To be used by Gazebo for the message passing between its server and client \\ \hline 
	\end{tabular}
	\caption{Design standards table.}
\end{table}\label{tab: design-standards}

\subsection{Professional code of ethics}

\blindtext

\subsection{Assumptions}

\blindtext

\end{document}
