\documentclass[../main.tex]{subfiles}
\graphicspath{{\subfix{../images/}}}

\begin{document}

\subsection{Functional requirements}

This project mainly focuses on two main settings: a simulated
environment and a real world experiment. 
Based on these, the different use cases are illustrated in
\cref{fig:use-case-diagram} and the following list:

\begin{enumerate}
    \item The users of the system must be authorized individuals and
        organizations to track, monitor or survey some moving objects.
    \item The user should train a new model.
    \item The user should test the model.
    \item The user must verify that the model is operating as expected.
    \item Model migration must occur from the simulated environment to the 
    real world.
    \item The user can start each mission, and start the moving targets.
\end{enumerate}

\noindent The details are summarized in \cref{tab:use-case-summaries}.

\begin{figure}[tbp] 
    \centering
    \includegraphics[width=0.9\textwidth]{newer-use-case-diagram} 
    \caption{The use case diagram for the system proposed 
            in this project depicting possible interactions 
            between the user and the system.}
    \label{fig:use-case-diagram} 
\end{figure}

\begin{center}
    \begin{xltabular}{\textwidth}{ p{3.5cm} X }
        \caption{Summary of the use cases of the system.}
        \label{tab:use-case-summaries} \\
        \toprule

        \textit{Use case} & \textit{Description} 
        \\

        \midrule
        \endfirsthead

        \caption[]{Summary of the use cases of the system (continued)} \\
        \toprule

        \textit{Use case} & \textit{Description} 
        \\

        \midrule
        \endhead

        Train a new model
          & A simulation environment with the correct targets
          appearance and behaviour (e.g. mobility pattern), and
          surroundings has to be set up. 
          Using this environment, an \gls{rl} and a \gls{ml} models
          have to be trained. 
          The former is to teach the drone what actions to take to
          accomplish the target visitation mission, and the latter is
          for the drone to uniquely identify the targets. 
        \\
        Test and verify the model
            & The trained model has to be tested on previously unseen
            movements of the targets generated from the same
            distribution as in the training.
        \\
        Migrate the model
            & After successful testing in the simulation, the drone
            must be tested in the real environment, and targets with
            similar outlook features and behaviour have to be created
            for this test.
        \\
        Start the drone
            & In the real mission, the drone should be started when
            the mobile targets are in a state similar to their initial
            states in the prior training and testing.
        \\

        \bottomrule
    \end{xltabular}
\end{center}
\vspace{-1.5cm}
%
\subsection{Design constraints}

The following lists in 
\cref{tab:technical-design-constraints,tab:practical-design-constraints} 
are the restrictions that have been imposed
on our component, technology and technique selections. 
They are relevant to our project 
because they have arisen due to the nature of 
the devices and technologies we have chosen to use,
which are namely the \anafi drone, the Raspberry Pi
and the reinforcement learning, as well as
the environment within which the drone will fly.

\begin{center}
    \begin{xltabular}{\textwidth}{ X p{12.2cm} }
        \caption{Technical design constraints.}
        \label{tab:technical-design-constraints} \\
        \toprule
        \textit{Name} 
            & \textit{Description} \\

        \midrule
        \endfirsthead

        \caption[]{Technical design constraints (continued)}\\
        \toprule
        \textit{Name} 
            & \textit{Description} \\

        \midrule
        \endhead

        Power supply  
            & The \anafi drone and the Raspberry Pi must be 
            battery powered because they are mobile 
            (\SI{2700}{\milli\ampere\hour} 
            for the \anafi drone and 
            \SI{4000}{\milli\ampere\hour} 
            for the Raspberry Pi)~\cite{Par19}.  \\

        \raggedright Altitude-cell size relationship 
            & There is a relationship between the size of the cells
            used to divide the grid on the ground and the height at
            which the drone needs to fly at.  In particular, the
            drone's height determines its \gls{fov} which needs to
            cover the cell underneath in its entirety but not too much
            of the surrounding cells. \\

        Flying duration
            & The maximum flying duration is 
            \SI{15}{minutes}
            as the battery is drained by 
            \SI{10}{\percent}
            on average in 
            \SI{1.5}{minutes} 
            when it is flown with a payload of 
            \SI{190}{grams}.
            This means that the target visitation
            mission needs to be completed within 
            this time 
            (the Raspberry Pi can last for 
            \SI{4}{hours} 
            under normal operations). \\ 

        Response time
            & The time between a command sent by 
            the computer, the \anafi drone 
            starting to execute it, and finally
            the computer receiving the confirmation
            of the execution must be
            below 
            \SI{1}{second}. 
            This is to facilitate real-time monitoring
            on the command and control station. \\

        Payload  
            & The Raspberry Pi with its peripherals 
            must not exceed 
            \SI{180}{grams}, 
            which is the maximum payload mass 
            of the \anafi drone. \\

        \gls{drl} model \newline accuracy
            & The most powerful computer at our disposal
            provides a dedicated \textsc{nvidia} 
            \textsc{titan} \textsc{8gb} \textsc{gpu}. 
            Even with these specifications, the training
            will last for an unreasonably long time 
            to allow the agent to exhaustively explore 
            the state space and find the global optimum.
            Therefore the accuracy of the \gls{drl} model will converge
            at a suboptimal level.
            Moreover, the sampling nature of \gls{rl}
            means that an optimal solution can never be
            guaranteed. \\

        \bottomrule
    \end{xltabular}
\end{center}
\vspace{-1.0cm}
%
% \begin{table}[H]
%     \centering
%     \caption{Practical design constraints.}
%     \label{tab:practical-design-constraints}
%     \begin{tabularx}{\textwidth}{ p{3cm} p{3cm} X }
%         \toprule
%         \textit{Type} 
%             & \textit{Name} 
%                 & \textit{Description} \\

%         \midrule
        
%         Economic 
%             & Cost 
%                 & The selling price of the entire 
%                 system must not exceed 
%                 \SI{6000}[\textsc{qar}\,]  
%                 or the product may not be sellable 
%                 due to being too expensive. \\
        
%         Environmental 
%             & Temperature 
%                 & The \anafi drone cannot be flown
%                 when the temperature outside is below
%                 \SI{-10}{\celsius}
%                 or above
%                 \SI{40}{\celsius}% 
%                 ~\cite{Par19}. \\

%         Environmental 
%             & Wind 
%                 & The \anafi drone cannot be flown
%                 when the wind speed is above 
%                 \SI{50}{\kilo\meter\per\hour}%
%                 ~\cite{Par19}. \\
        
%         Deployment 
%             & \gls{drl} model 
%                 & The software system must be such 
%                 that when the details of the task change,
%                 it should be easy for the \gls{drl} model 
%                 to be swapped with another one trained for
%                 the new task. \\

%         Environmental 
%             & Eco-friendliness 
%                 & The system only uses electricity 
%                 and no emission. \\
        
%         \bottomrule		
%     \end{tabularx}
% \end{table}
%
\begin{table}[H]
    \centering
    \caption{Practical design constraints.}
    \label{tab:practical-design-constraints}
    \begin{tabularx}{\textwidth}{ c p{4cm} X }
        \toprule
        \textit{No.} 
            & \textit{Constraint} 
                & \textit{Relevance to the design} \\

        \midrule
        
        1 
            & Public Health, Safety, and Welfare 
                & %
                \hspace{-0.7cm} \begin{minipage} [t] {0.65\textwidth} 
                \begin{itemize}
                  \item The \anafi drone cannot be flown
                    when the temperature outside is below
                    \SI{-10}{\celsius}
                    or above
                    \SI{40}{\celsius}% 
                    ~\cite{Par19}. \vspace{-0.2cm}
                  \item The \anafi drone cannot be flown
                    when the wind speed is above 
                    \SI{50}{\kilo\meter\per\hour}%
                    ~\cite{Par19}.
               \end{itemize} 
               \end{minipage}
                \\ \addlinespace
        
        2 
            & Global Implications 
                & The product can lead to an increase in the
                exploration of marine biology, and aid in monitoring
                and protection of rare animals from extinction 
                \\ \addlinespace

        3 
            & Cultural Implications 
                & N/A \\
                \addlinespace
        
        4 
            & Social Implications
                & %
                \hspace{-0.7cm} \begin{minipage} [t] {0.65\textwidth} 
                \begin{itemize}
                  \item The autonomous drone can help track and catch
                      criminals in the shortest possible time.%
                      \vspace{-0.2cm}
                  \item Consequently, it will improve the safety and
                      prosperity of the society.
               \end{itemize} 
               \end{minipage}
                \\ \addlinespace
        5 
            & Environmental Factors
                & The system only uses electricity 
                and no emission. \\
                \addlinespace
        
        6 
            & Economic Factors
                & The selling price of the entire 
                system must not exceed 
                \SI{6000}[\textsc{qar}\,]  
                or the product may not be sellable 
                due to being too expensive. \\
                \addlinespace
        
        \bottomrule		
    \end{tabularx}
\end{table}
\subsection{Design standards}

The use of standards ensures reliability and consistency
among the devices and programs that incorporate them.
The standards employed in the design of this project are
listed in the following \cref{tab:design-standards}.

\begin{table}[H]
    \centering
    \caption{Design standards table.}
    \label{tab:design-standards}
    \begin{tabularx}{\textwidth}{ X p{12.3cm} }
        \toprule
            \textit{Standard} 
                & \textit{Usage}\\

        \midrule
        \gls{ieee} 802.11 
                & Used in the communication between 
                the Raspberry Pi and the \anafi drone,
                and between the 
                Raspberry Pi and the computer. \\ 
                \addlinespace
        
        \textsc{wpa}2 
                & Used in securing the 
                communications between the drone and the command and
                control system. \\
                \addlinespace
        
        \textsc{gps}  
                & Used by the \anafi drone to 
                convey its position 
                to the Raspberry Pi. \\
                \addlinespace
        
        \textsc{ssh} 
                & Used by the computer to 
                send high-level commands to the 
                Raspberry Pi, which in turn, makes
                the drone execute them. \\
                \addlinespace
        
        \textsc{json-rpc} 2.0 
                & Used to control the components 
                of the simulated \anafi drone. \\
        
        \bottomrule
    \end{tabularx}
\end{table}

\subsection{Professional code of ethics}

\noindent
In accordance with the \gls{ieee} Code of Ethics:
\begin{itemize}
    \item[I-7] Stakeholders and users 
        should accept our terms and services which include 
        avoiding injuring others and their property 
        and protecting the privacy of others.
    \item[II-7] Stakeholders and users should not use our product 
        or service to discriminate any
        part of the community according 
        to their race, religion, etc.
\end{itemize}

\noindent
In accordance with the \gls{acm} Code of Ethics:
\begin{itemize}
    \item[1.6] Respect privacy of the people and their 
        properties and collect/train data after confirmations.
    \item[1.7] The client's data and information, which is gathered 
        to make the product consisting of the model along with 
        the drone kit, should be kept confidential except in cases 
        where the application violates the law, 
        organizational regulations or the Code.
    \item[2.3] Usage of product/services should be under 
        the local and international laws and regulations. 
        For example, you need a license to fly a drone in Qatar.
\end{itemize}

\subsection{Assumptions}

The following statements are assumed to be true
since the design decisions to meet the opposite
conditions fall outside the scope of this project.

\begin{enumerate}
    \item The environment is obstacle-free with little 
        to no wind.
    \item The site is bright enough for the drone to
        detect and identify the targets.
    \item The real targets move according to the mobility 
        pattern used to train the machine learning model.
    \item The targets in simulation and practice are 
        equally detectable and identifiable using 
        machine learning.
    \item The \anafi drone captures the 
        required details of the target 
        when it takes the target’s picture.
    \item The \anafi drone is fully charged before 
        initiating each target visitation task,
        which is completed before the drone
        runs out of battery.
    \item The security mechanisms used by the \anafi
        drone are enough to protect against
        any cyber and physical attacks.
\end{enumerate}

\end{document}
