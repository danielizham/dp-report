\documentclass[../main.tex]{subfiles}

\begin{document}

\subsection{Functional requirements}

In order to achieve the objectives listed
in \cref{sec:objectives}, the following activities 
and operations should be expected by the user 
and the \gls{uav} system:

\begin{enumerate}
    \item The users of the system are authorized
        individuals and organizations who
        need to track, monitor or survey
        some moving objects.
    \item The user should supply the 
        profile of the targets 
        and their movement details/patterns
        to us for the \gls{rl}.
    \item The user should receive 
        an \anafi-Raspberry Pi-\gls{drl} 
        model bundle after a maximum of 3 months.
    \item The system should take pictures, capture videos
        or record other physical properties using
        suitable sensors of the intended
        targets according to the needs
        of the user.
    \item The user should receive some 
        guidance and support 
        to handle the system.
    \item The user should agree on 
        the terms and conditions 
        which limit the type of users.
    \item The system should only allow 
        authorized users 
        to view the collected pictures and 
        data from the \anafi system.
    \item The users must not be allowed to access 
        the source code.
\end{enumerate}

\subsection{Design constraints}

\begin{table}[H]
    \centering
    \caption{Technical design constraints.}
    \label{tab:technical-design-constraints}
    \begin{tabular}{ P{2.8cm} P{12.2cm} }
        \toprule
        \textit{Name} 
            & \textit{Description} \\

        \midrule

        Power supply  
            & The \anafi drone and the Raspberry Pi must be 
            battery powered because they are mobile 
            (2700mAh for the \anafi drone 
            and 4000mAh for the Raspberry Pi)  \\

        Flying range/altitude 
            & The flying range of the \anafi drone is 
            limited 
            by the signal strength of the communication 
            between the computer and the Raspberry Pi. 
            Also, the altitude is dictated 
            by the \anafi drone’s 
            maximum height it can reach 
            and the performance 
            of the drone at a certain height. \\

        Flying time  
            & The maximum flying time must be lower than 
            the \anafi drone’s maximum capability 
            which is 25 minutes 
            (the Raspberry Pi can last for 4 hours under normal
            operations) \\ 

        Transmission delay  
            & The delay in the communication between 
            the Raspberry Pi and the \anafi drone
            must be acceptable \\

        Payload/weights  
            & The Raspberry Pi with its peripherals should not 
            exceed the maximum load the \anafi 
            drone can carry. \\

        \bottomrule
    \end{tabular}
\end{table}

\begin{table}[H]
    \centering
    \caption{Practical design constraints.}
    \label{tab:practical-design-constraints}
    \begin{tabular}{ P{3cm} P{3cm} P{9cm} }
        \toprule
        \textit{Type} 
            & \textit{Name} 
                & \textit{Description} \\

        \midrule
        
        Economic 
            & Cost 
                & The selling price of the entire system must 
                not exceed \textsc{qar} 6000 or the product
                may not be sellable due to being too expensive \\
        
        Environmental 
            & Light source 
                & The \anafi drone requires 
                good lighting to 
                take pictures of the targets \\
        
        Security 
            & Communication 
                & The communication between \anafi-Raspberry Pi 
                and Raspberry Pi-computer should be uninterrupted 
                and secured against interference \\
        
        Safety 
            & Safeguard 
                & The \anafi drone must know how 
                to return to base 
                should there be a discommunication with the computer \\
        
        Ethical 
            & Privacy 
                & The drone must not invade the privacy of 
                the entities other than the targets \\
        
        Ethical 
            & Law-abiding 
                & The monitored targets must be acceptable 
                by law to be tracked \\
        
        Environmental 
            & Eco-friendly 
                & The system only uses electricity and emissions \\
        
        Sustainability 
            & Modularity 
                & The software system must be such that the \gls{drl} model 
                can be swapped with an improved one easily \\
        
        Reliability 
            & Efficiency 
                & The \gls{drl} should perform as expected all the time \\

        \bottomrule		
    \end{tabular}
\end{table}

\subsection{Design standards}

\begin{table}[H]
    \centering
    \caption{Design standards table.}
    \label{tab:design-standards}
    \begin{tabular}{ P{2.2cm} P{12.3cm} }
        \toprule
            \textit{Standard} 
                & \textit{Usage}\\

        \midrule
        \gls{ieee} 802.11 
                & To be used in the communication between 
                the Raspberry Pi and the \anafi drone
                and the 
                Raspberry Pi and the computer \\ 
                \addlinespace
        
        \textsc{wpa}2 
                & To be used in securing the communications above \\
                \addlinespace
        
        
        \textsc{gps}  
                & To be used by the \anafi drone to 
                convey its position 
                to the Raspberry Pi \\
                \addlinespace
        
        
        \textsc{ssh} 
                & To be used between the Raspberry Pi and the computer \\
                \addlinespace
        
        
        \textsc{json-rpc} version 2.0 
                & Communication protocol used to control the components 
                of the simulated \anafi drone \\
                \addlinespace
        
        
        Google Protocol Buffers 
                & To be used by Gazebo for the message passing between 
                its server and client \\
        
        \bottomrule
    \end{tabular}
\end{table}

\subsection{Professional code of ethics}

\noindent
In accordance with the \gls{ieee} Code of Ethics:
\begin{itemize}
    \item[I-7] Stakeholders and users 
        should accept our terms and services which include 
        avoiding injuring others and their property 
        and protecting the privacy of others.
    \item[II-7] Stakeholders and users should not use our product 
        or service to discriminate any
        part of the community according 
        to their race, religion, etc.
\end{itemize}

\noindent
In accordance with the \gls{acm} Code of Ethics:
\begin{itemize}
    \item[1.6] Respect privacy of the people and their 
        properties and collect/train data after confirmations.
    \item[1.7] The client's data and information, which is gathered 
        to make the product consisting of the model along with 
        the drone kit, should be kept confidential except in cases 
        where the application violates the law, 
        organizational regulations or the Code.
    \item[2.3] Usage of product/services should be under 
        the local and international laws and regulations. 
        For example, you need a license to fly a drone in Qatar.
\end{itemize}

\subsection{Assumptions}

\begin{enumerate}
    \item Obstacle-free environment
    \item Windless environment
    \item The targets move according to one of the mobility patterns
    \item The physical \anafi drone performs identically to the simulated \anafi drone
    \item The \anafi drone captures the required/full details of the target 
        when it takes the target’s picture
    \item Similarly, when it takes the target’s picture, 
        it is assumed the task at hand 
        (e.g. criminal tracking, smart farming, search and rescue, etc.) 
        is accomplished.
\end{enumerate}

\end{document}
